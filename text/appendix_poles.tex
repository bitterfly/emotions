\documentclass[main.tex]{subfiles}
\begin{document}
\chapter{Приложение за полюси и нули}
\label{appendix:poles}
\section{Дефиниция}

Нека $\mathcal{z} \in \mathbb{C}$. Видяхме,че предавателната функция $\mathcal{H}$ на определени системи (и в частност филтри) има вида:
\begin{flalign}
    \tag{\ref{eq:systems:6}}
    & \mathcal{H}(\mathcal{z}) =  \cfrac{\sum\limits_{m=0}^{M} b_m \mathcal{z}^{-m}}{\sum\limits_{k=0}^{N} a_k \mathcal{z}^{-k}} = && \\
    \nonumber & = \cfrac{N(\mathcal{z})}{D(\mathcal{z})} = G \cfrac{(\mathcal{z} - \beta_1)(\mathcal{z} - \beta_2)...(\mathcal{z} - \beta_M)}{(\mathcal{z} - \alpha_1)(\mathcal{z} - \alpha_2)...(\mathcal{z} - \alpha_N)}, &&
\end{flalign}

където $G = b_0/a_0$ и се нарича усилващ коефициент.

С $\beta_i$ означаваме корените на уравнението $N(\mathcal{z}) = 0$. Те се наричат нули на системата и 
$\lim\limits_{\mathcal{z} \rightarrow \beta_i} \mathcal{H}(\mathcal{z}) = 0$

С $\alpha_i$ означаваме корените на уравнението $D\mathcal{z}(\mathcal{z}) = 0$. Те се наричат полюси на системата и 
$\lim\limits_{\mathcal{z} \rightarrow \alpha_i} \mathcal{H}(\mathcal{z}) = \infty$

Тъй като коефициентите на $N(\mathcal{z})$ и $D(\mathcal{z})$ са реални, нулите (и съответно полюсите) ще са
са или реални, или са част от двойка комплексно спрегнати. Тоест, няма нула (или полюс), която да е комплексна, но
да няма комплексно спрегнато из останалите нули (полюси).

Това представяне е удобно, защото ни позволява да разбием $\mathcal{H}$ на произведение от по-прости предавателни функции:

\begin{flalign*}
    & \mathcal{H}(\mathcal{z}) = 
        \underbrace{G \frac{(\mathcal{z} - \beta_1)}{(\mathcal{z} - \alpha_1)}}_{} 
        \underbrace{\frac{(\mathcal{z} - \beta_2)}{(\mathcal{z} - \alpha_2)}}_{} 
        \dots
        \underbrace{\frac{(\mathcal{z} - \beta_M)}{(\mathcal{z} - \alpha_N)}}_{} &&\\
    & \mathcal{H}(\mathcal{z}) = \mathcal{H}_1(\mathcal{z})\mathcal{H}_2(\mathcal{z})\dots\mathcal{H}_K(\mathcal{z}),&&
\end{flalign*}
където $\mathcal{H}_i$ е произведение на няколко полюса и нули.

Тоест, достатъчно е да видим какви филтри се описват от трансферни функции, съдържащи една или две нули и полюси, за да можем да направим извод за целия филтър $\mathcal{H}$.


\begin{figure}[ht]
\centering
\begin{tikzpicture}
    \begin{axis}[
        xlabel={реална ос},
        ylabel={имагинерна ос},
        xmin=-1.2, xmax=1.2,
        ymin=-1.2, ymax=1.2,
        xtick={-1, -0.5, 0, 0.5, 1},
        ytick={-1, -0.5, 0, 0.5, 1},
        ymajorgrids=true,
        xmajorgrids=true,
        grid style=dashed,
        axis equal image, 
        legend pos=outer north east,
        legend cell align={left},
        legend entries={
            полюс,
            нула
        },
    ]

    \addlegendimage{only marks, mark=square,color=red, thick}
    \addlegendimage{only marks,mark=o, color=blue, thick}

    \draw [red, thick] (axis cs:0.75, -0.05) rectangle (axis cs:0.85, 0.05);
    \draw [red, thick] (axis cs:0.41, -0.05) rectangle (axis cs:0.51, 0.05);
    \draw [red, thick] (axis cs:0.70, 0.95) rectangle (axis cs:0.80, 1.05);
    \draw [red, thick] (axis cs:0.70, -0.95) rectangle (axis cs:0.80, -1.05);
    \draw [blue, thick] (axis cs:-0.5, 0) circle [radius=6];
    \draw [blue, thick] (axis cs:-1, -0.7) circle [radius=6];
    \draw [blue, thick] (axis cs:-1, 0.7) circle [radius=6];
    \draw[black,thick,dashed] (axis cs:0,0) circle [radius=100];
    \end{axis}

    \end{tikzpicture}
    \caption{Полюс-нула графика} \label{fig:appendix:1:1:1}
\end{figure}

\autoref{fig:appendix:1:1:1} изобразява трансферна функция с три нули и четири полюса, 
от които една реална нула и два реални полюса. Нулите и полюсите, които не са реални, са
комплексно спрегнати.

\section{Характеризация на филтри}

Една система се описва изцяло от трансферната си функция, а всяка трансферна функция може да се представи като произведение на нули и полюси. Следователно, анализирайки тези нули и полюси, можем да направим извод за действието на филтъра.

Нека имаме следния филтър от първи ред:
\begin{flalign*}
    & y[n] = b_0 x[n] + a_1 y[n-1] \xleftrightarrow{\mathcal{F}\mathcal{S}} && \\
    \\
    & Y(\mathcal{z}) = b_0 X(\mathcal{z}) + a_1 \mathcal{z}^{-1} Y(\mathcal{z}) \longleftrightarrow && \\
    \\
    & \cfrac{Y(\mathcal{z})}{X(\mathcal{z})} = \cfrac{b_0}{1 - a_1\mathcal{z}^{-1}} \longleftrightarrow && \\
    \\
    & \mathcal{H}(\mathcal{z}) = G \frac{1}{1 - a_1\mathcal{z}^{-1}}, G = b_0 &&
\end{flalign*}

Тъй като имаме само един полюс, то следва, че $a_1$ е реално число, тъй като няма как да е част от
комплексно спрегната двойка. Това означава, че $a_1$ напълно описва вида на $\mathcal{H}$, а
$b_0$ играе ролята на усилващ коефициент.

\begin{figure}[H]%
    \centering
        \subfloat[Полюс-нула графика за $\mathcal{H}$]{%
        \begin{tikzpicture}[baseline]
            \begin{axis}[
                xlabel={реална ос},
                ylabel={имагинерна ос},
                xmin=-1.2, xmax=1.2,
                ymin=-1.2, ymax=1.2,
                xtick={-1, -0.5, 0, 0.5, 1},
                ytick={-1, -0.5, 0, 0.5, 1},
                ymajorgrids=true,
                xmajorgrids=true,
                grid style=dashed,
                axis equal image, 
            ]
            \draw [red, thick] (axis cs:0.75, -0.05) rectangle (axis cs:0.85, 0.05);
            \draw [red, thick] (axis cs:0.45, -0.05) rectangle (axis cs:0.55, 0.05);
            \draw [red, thick] (axis cs:0.30, -0.05) rectangle (axis cs:0.40, 0.05);
            \draw[black,thick,dashed] (axis cs:0,0) circle [radius=100];
            \end{axis}
        \end{tikzpicture}
        } 
        \hfill
        \subfloat[Графика на $g(\omega)$. От горе надолу:$a=0.7, a=0.5, a=0.35$]{%
        \begin{tikzpicture}[baseline]
            \begin{axis}[
                xlabel={Ъглова честота},
                ylabel={Амплитуда},
                xmin=-4, xmax=4,
                ymin=0, ymax=5.5,
                xtick={-4, -3, -2, -1, 0, 1, 2, 3, 5.5},
                ytick={0, 1, 2, 3, 4, 5, 6},
                ymajorgrids=true,
                xmajorgrids=true,
                grid style=dashed,]
                \addplot [domain=-pi:pi, samples=500,]{(1/(1 - 2*0.7*cos(deg(x)) + (0.7)^2))^1/2};
                \addplot [domain=-pi:pi, samples=500, dotted, thick]{(1/(1 - 2*0.5*cos(deg(x)) + (0.5)^2))^1/2};
                \addplot [domain=-pi:pi, samples=500, dotted, thick]{(1/(1 - 2*0.35*cos(deg(x)) + (0.35)^2))^1/2};
                \end{axis}
        \end{tikzpicture}
        }
        \caption{Действие на филтър от първи ред за $a=0.35, a = 0.5, a = 0.7$}
        \label{fig:appendix:1:2:1}
\end{figure}   
\begin{figure}[H]%
    \centering
    \subfloat{
        \begin{tikzpicture}[baseline]
            \begin{axis}[
                xlabel={Ъглова честота},
                ylabel={Амплитуда},
                xmin=0, xmax=pi,
                ymin=0, ymax=5.5,
                xtick={0,2.5, 5},
                ytick={1, 2, 3, 4, 5, 6},
                ymajorgrids=true,
                xmajorgrids=true,
                grid style=dashed,]
                \addplot [domain=-pi:pi, samples=500, thick, blue]{(1/(1 - 2*-0.7*cos(deg(x)) + (-0.7)^2))^1/2};
                \addplot [domain=-pi:pi, samples=500, thick]{(1/(1 - 2*0.7*cos(deg(x)) + (0.7)^2))^1/2};
                \addplot [domain=-pi:pi, samples=500, thick, dotted]{(2.8/(1 - 2*0.5*cos(deg(x)) + (0.5)^2))^1/2};
                \addplot [domain=-pi:pi, samples=500, thick, dotted]{(0.2/(1 - 2*0.865*cos(deg(x)) + (0.865)^2))^1/2};
            \end{axis}
        \end{tikzpicture}
        }
        \caption{Действие на филтър от първи ред за различни стойности на $a$ и $b$\\С черно от ляво надясно:\\a = 0.5, b = 2.8;\\a = 0.7, b = 1;\\a = 0.865, b = 0.2\\Със синьо:\\a = -0.7, b = 1}
        \label{fig:appendix:1:2:2}
\end{figure}   

Понеже $a_1$ е реално число, винаги ще лежи на реалната ос, както е показано на полюс-нула графиката на \autoref{fig:appendix:1:2:2}

Нека разгледаме $\mathcal{H}$ в честотния домейн:
$\mathcal{H}(e^{\mathcal{i}\omega}) = \cfrac{b_0}{1 - a_1e^{-\mathcal{i}\omega}},$ където $\omega$ е ъглова честота, измерена в радиани.
Можем изразим $\mathcal{H}$ като функция на $\omega$:

\begin{flalign*}
    & \mathcal{H}(e^{\mathcal{i}\omega}) = \cfrac{b_0}{1 - a_1e^{-\mathcal{i}\omega}} = \cfrac{b_0}{1 - a_1 \cos{\omega} + \mathcal{i} a_1 \sin{\omega}} 
    = \cfrac{b_0(1 - a_1\cos{\omega} - \mathcal{i}a_1 \sin{\omega} )}{1 - 2 a_1 \cos{\omega} + a_1 ^ 2} && \\ 
    & = \frac{b_0(1 - a_1 \cos{\omega})}{{1 - 2 a_1 \cos{\omega} + a_1 ^ 2}} + \mathcal{i}\cfrac{- b_0 a_1 \sin{\omega}}{{1 - 2 a_1 \cos{\omega} + a_1 ^ 2}}&&
\end{flalign*}
Нека с $g(\omega)$ означим модула на $\mathcal{H}$
\begin{flalign*}
    & g(\omega) = \cfrac{b_0^2(1 - 2 a_1 \cos{\omega} + a_1^2)}{(1 - 2 a_1 \cos{\omega} + a_1 ^ 2)^2} = \cfrac{b_0^2}{1 - 2 a_1 \cos{\omega} + a_1 ^ 2} &&
\end{flalign*}
На \autoref{fig:appendix:1:2:2} се вижда графиката на $g(\omega)$ за различни стойности на $a_1$ и $b_0$.\\
Този вид филтри се наричат \textbf{резонатори}, тъй като честотите във върха на графиката ще се усилят.
Резонаторите се описват главно чрез своята \textbf{амплитуда} - височината на максимума, \textbf{честота} - къде е върхът върху честотната ос, 
\textbf{честотна лента} - колко е широка графиката, което определя колко честоти ще се усилят.

В случая на филтър от първи ред, амплитудата и честотната лента се определят от $a_1$ и $b_0$,
а върха на графиката винаги ще е в 0. Тоест този вид филтри могат да усилват само честотите около 0. 

При $a_1 > 0$, филтрите се наричат \textbf{нискочестотни}, защото пропускат ниските честоти и задържат високите (с черно на \autoref{fig:appendix:1:2:2}).

При $a_1 < 0$, филтрите се наричат \textbf{високочестотни} (в синьо на \autoref{fig:appendix:1:2:2})

За да се премести пикът на функцията нанякъде по честотната ос извън нулата, трябва $a_1$ да е комплексно. Ако трансферната функция има само един полюс, $a_1$ винаги е реално,
затова ни трябва поне една комплексно спрегната двойка. Нека разгледаме филтър от втори ред.

\begin{flalign*}
    &y[n] = b_0x[n] + a_1 y[n-1] + a_2 y[n-2] && \\
    &\mathcal{H}(\mathcal{z}) = \cfrac{b_0}{1 - a_1\mathcal{z}^{-1} - a_2\mathcal{z}^{-2}} && \\
    &\mathcal{H}(\mathcal{z}) = G \cfrac{1}{(1 - \alpha_1 {z}^{-1})(1 - \alpha_2 {z}^{-1})} &&
\end{flalign*}

\begin{figure}[H]%
    \centering
        \subfloat[Полюс-нула графика за $\mathcal{H}$]{%
        \begin{tikzpicture}[baseline]
            \begin{axis}[
                xlabel={реална ос},
                ylabel={имагинерна ос},
                xmin=-1.2, xmax=1.2,
                ymin=-1.2, ymax=1.2,
                xtick={-1, -0.5, 0, 0.5, 1},
                ytick={-1, -0.5, 0, 0.5, 1},
                ymajorgrids=true,
                xmajorgrids=true,
                grid style=dashed,
                axis equal image, 
            ]
            \draw [red, thick] (axis cs:0.20, 0.45) rectangle (axis cs:0.30, 0.55);
            \draw [red, thick] (axis cs:0.20, -0.45) rectangle (axis cs:0.30, -0.55);
            \draw [red, thick] (axis cs:0.45, 0.70) rectangle (axis cs:0.55, 0.80);
            \draw [red, thick] (axis cs:0.45, -0.70) rectangle (axis cs:0.55, -0.80);
            \draw [red, thick] (axis cs:0.325, 0.575) rectangle (axis cs:0.425, 0.675);
            \draw [red, thick] (axis cs:0.325, -0.575) rectangle (axis cs:0.425, -0.675);
            \draw[black,thick,dashed] (axis cs:0,0) circle [radius=100];
            \end{axis}
        \end{tikzpicture}
        }\hfill
        \subfloat[Графика на $g(\omega)$ в $-\pi, \pi$]{%
        \begin{tikzpicture}[baseline]
            \begin{axis}[
                xlabel={Ъглова честота},
                ylabel={Амплитуда},
                xmin=-4, xmax=4,
                ymin=0, ymax=5.5,
                xtick={-4, -3, -2, -1, 0, 1, 2, 3, 5.5},
                ytick={0, 1, 2, 3, 4, 5, 6},
                ymajorgrids=true,
                xmajorgrids=true,
                grid style=dashed,]
                \addplot [domain=-pi:pi, samples=500,]{( (1 - (1.6)*cos(deg(x)) - (-3.2)*cos(deg(2*x)))^2 + ((1.6)*sin(deg(x)) + (-3.2)*sin(deg(2*x)) )^2  )^(-1/2)};
                \addplot [domain=-pi:pi, samples=500,dotted, thick]{( (1 - (1.23077)*cos(deg(x)) - (-1.23077)*cos(deg(2*x)))^2 + ((1.23077)*sin(deg(x)) + (-1.23077)*sin(deg(2*x)) )^2  )^(-1/2)};
                \addplot [domain=-pi:pi, samples=500,dotted, thick]{( (1 - (1.41176)*cos(deg(x)) - (-1.88235)*cos(deg(2*x)))^2 + ((1.41176)*sin(deg(x)) + (-1.88235)*sin(deg(2*x)) )^2  )^(-1/2)};
                \end{axis}
        \end{tikzpicture}
        }
        \caption{Действие на филтър от втори ред за $\alpha_1 = (0.25 + 0.5i), (0.5 + 0.75i), (0.375 + 0.625)$}
        \label{fig:appendix:1:2:3}
\end{figure}  

\begin{figure}[H]%
    \centering
    \subfloat{%
        \begin{tikzpicture}[baseline]
            \begin{axis}[
                xlabel={Ъглова честота},
                ylabel={Амплитуда},
                xmin=0, xmax=4,
                ymin=0, ymax=5,
                xtick={0, 2.5, 5},
                ytick={0, 1, 2, 3, 4},
                ymajorgrids=true,
                xmajorgrids=true,
                grid style=dashed,]
                \addplot [domain=-pi:pi, samples=500, thick]{( (1 - (1.17)*cos(deg(x)) - (-0.64)*cos(deg(2*x)))^2 + ((1.17)*sin(deg(x)) + (-0.64)*sin(deg(2*x)) )^2  )^(-1/2)};
                \addplot [domain=-pi:pi, samples=500, thick, red]{( (1 - (0.86)*cos(deg(x)) - (-0.64)*cos(deg(2*x)))^2 + ((0.86)*sin(deg(x)) + (-0.64)*sin(deg(2*x)) )^2  )^(-1/2)};
                \addplot [domain=-pi:pi, samples=500, thick, blue]{( (1 - (0.50)*cos(deg(x)) - (-0.64)*cos(deg(2*x)))^2 + ((0.50)*sin(deg(x)) + (-0.64)*sin(deg(2*x)) )^2  )^(-1/2)};
                \addplot [domain=-pi:pi, samples=500, thick, green]{( (1 - (0.11)*cos(deg(x)) - (-0.64)*cos(deg(2*x)))^2 + ((0.11)*sin(deg(x)) + (-0.64)*sin(deg(2*x)) )^2  )^(-1/2)};
                \end{axis}
        \end{tikzpicture}
    }
    \caption{Графика на $g$ с отдалечаващи се от реалната ос полюси}
    \label{fig:appendix:1:2:4}
\end{figure} 

\begin{figure}[H]
    \centering
    \subfloat[$b_0 = 1, b_1 = 0$]{%
    \begin{tikzpicture}[baseline]
        \begin{axis}[
            xlabel={Ъглова честота},
            ylabel={Амплитуда},
            xmin=-4, xmax=4,
            ymin=0, ymax=4.5,
            xtick={-4, -3, -2, -1, 0, 1, 2, 3, 4},
            ytick={0, 1, 2, 3, 4},
            ymajorgrids=true,
            xmajorgrids=true,
            grid style=dashed,]
            \addplot [domain=-pi:pi, samples=500, thick]{((1/( ((1.17)*sin(deg(x)) + (-0.64)*sin(deg(2*x)))^2 + (-(1.17)*cos(deg(x)) -(-0.64)*cos(deg(2*x)) + 1)^2 ))^(1/2)};
        \end{axis}
    \end{tikzpicture}
    }\vfill
    \subfloat[$b_0 = 1, b_1 = 1$]{%
        \begin{tikzpicture}[baseline]
        \begin{axis}[
            xlabel={Ъглова честота},
            ylabel={Амплитуда},
            xmin=-4, xmax=4,
            ymin=0, ymax=3.5,
            xtick={-4, -3, -2, -1, 0, 1, 2, 3, 4},
            ytick={0, 1, 2, 3, 3.5},
            ymajorgrids=true,
            xmajorgrids=true,
            grid style=dashed,]
            \addplot [domain=-pi:pi, samples=500, thick]{(( (1 - 1*cos(deg(x)))^2 + 1^2*(sin(deg(x)))^2 )/( ((1.17)*sin(deg(x)) + (-0.64)*sin(deg(2*x)))^2 + (-(1.17)*cos(deg(x)) -(-0.64)*cos(deg(2*x)) + 1)^2 ))^(1/2)};
        \end{axis}
        \end{tikzpicture}
        }\hfill
        \subfloat[$b_0 = 1, b_1 = -1$]{%
        \begin{tikzpicture}[baseline]    
        \begin{axis}[
            xlabel={Ъглова честота},
            ylabel={Амплитуда},
            xmin=-4, xmax=4,
            ymin=0, ymax=8.8,
            xtick={-4, -3, -2, -1, 0, 1, 2, 3, 4},
            ytick={0, 2, 4, 6, 7.6, 8.8},
            ymajorgrids=true,
            xmajorgrids=true,
            grid style=dashed,]
            \addplot [domain=-pi:pi, samples=500, thick]{(( (1 - (-1)*cos(deg(x)))^2 + (-1)^2*(sin(deg(x)))^2 )/( ((1.17)*sin(deg(x)) + (-0.64)*sin(deg(2*x)))^2 + (-(1.17)*cos(deg(x)) -(-0.64)*cos(deg(2*x)) + 1)^2 ))^(1/2)};
        \end{axis}
        \end{tikzpicture}
        }
        \caption{Действие на филтър от втори ред за различни стойности на $b$ и $a_1 = 1.17, a_2 = -0.64$}
        \label{fig:appendix:1:2:5}
\end{figure}

Местенето на полюсите по-далеч от реалната ос, раздалечава върховете по честотната лента, както се вижда на \autoref{fig:appendix:1:2:4}

Видът на резонатора (тоест честотна лента, честота и амплитуда), се определят главно от полюсите.
Добавянето на нули също влияе на вида на филтъра, както може да се види от \autoref{fig:appendix:1:2:5}
В единият случай се добавя нула в нулата, в другия - в края на спектъра. 

\begin{figure}[H]
    \centering
    \subfloat{%
    \begin{tikzpicture}[baseline]
        \begin{axis}[
            xlabel={Ъглова честота},
            ylabel={Амплитуда},
            xmin=-4, xmax=4,
            ymin=0, ymax=5.5,
            xtick={-4, -3, -2, -1, 0, 1, 2, 3, 4},
            ytick={0, 1, 2, 3, 4, 5},
            ymajorgrids=true,
            xmajorgrids=true,
            grid style=dashed,
            legend cell align={left},
            legend entries={$b_0 = 1;b_1 = 1;b_2 = -1; b_3 = 0$,
            $b_0 = 1;b_1 = 1;b_2 = -1; b_3 = 1$,},
            legend pos=outer north east,]

            \addlegendimage{no markers,red}
            \addlegendimage{no markers,black}
        
            \addplot [domain=-pi:pi, samples=500, thick]{((( ((1)*sin(deg(x)) + (-1)*sin(deg(2*x)))^2 + (-(1)*cos(deg(x)) - (-1)*cos(deg(2*x)) + 1)^2 )/( ((1.17)*sin(deg(x)) + (-0.64)*sin(deg(2*x)))^2 + (-(1.17)*cos(deg(x)) -(-0.64)*cos(deg(2*x)) + 1)^2 ))^(1/2)};
            \addplot [domain=-pi:pi, samples=500, thick, red]{((( ((1)*sin(deg(x)) + (-1)*sin(deg(2*x))  -(1)*sin(deg(3*x)))^2 + (-(1)*cos(deg(x)) - (-1)*cos(deg(2*x)) + (1)*cos(deg(3*x)) + 1)^2 )/( ((1.17)*sin(deg(x)) + (-0.64)*sin(deg(2*x)))^2 + (-(1.17)*cos(deg(x)) -(-0.64)*cos(deg(2*x)) + 1)^2 ))^(1/2)};
        \end{axis}
    \end{tikzpicture}
    }
    \caption{Действие на филтър от вида $\mathcal{H}(e^{i \omega}) = \frac{b_0 - b_1 e^{-i \omega} -b_2 e^{-2 i \omega} -b_3 e^{-3 i \omega}}{1 - a_1e^{-i \omega} - a_2e^{-2i \omega}}$ за $a_1 = 1.17, a_2 = -0.64$}
    \label{fig:appendix:1:2:6}
\end{figure}

Добавянето на допълнителни нули може да се види на \autoref{fig:appendix:1:2:6}. Тези нули се наричат \textbf{антирезонанси}.

Тогава можем да разложим даден сложен филтър $\mathcal{H}$ по следния начин: 
\begin{flalign*}
    & \mathcal{H}(\mathcal{z}) = \mathcal{H}_1(\mathcal{z})\mathcal{H}_2(\mathcal{z})\dots\mathcal{H}_K(\mathcal{z}),&&
\end{flalign*}

където $H_i$ е по-прост филтър от първи или втори ред, чийто вид може лесно да се моделира чрез промяна на коефициентите.

След това съчетаването на простите е просто произведение в честотния домейн, а свойствата на Фурие
преобразуванията ни дават вида и във времевия домейн.
\end{document}

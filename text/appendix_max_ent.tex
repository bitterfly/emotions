e\documentclass[main.tex]{subfiles}
\begin{document}
\chapter{Приложение за Максимизиране на ентропията}
\label{appendix:max_ent}

Нека имаме входни данни $X\times Y = (x_1, y_1),\ldots, (x_n, y_n)$, където $x_i \in X$, а $y_i \in Y$. 

Това търсим разпределение $p$, което приближава разпределението, генерирало данните в $X\times Y$ и което се държи равномерно иначе.

Тоест търсеното разпределение $p$, трябва да изпълнява:
\begin{flalign*}
	& p(x, y) = \tilde{p}(x) p(y|x), &&
\end{flalign*}
където с $\tilde{p}$ означаваме емпиричната разпределение $\B{\forall(x, y) \in X\times Y: \tilde{p}(x, y) = \cfrac{\#(x, y)}{n}}.$ С други думи:
\begin{flalign*}
	p(x) = \sum\limits_{y\in Y} p(x, y) & = \sum\limits_{y \in Y} \tilde{p}(x) p(y|x)&&\\
	&= \tilde{p}(x)\sum\limits_{y\in Y} p(y|x)&&\\& = \tilde{p}(x)
\end{flalign*}

и да максимизира ентропията:
\begin{flalign*}
	H(X, Y) & = - \sum\limits_{(x, y) \in X\times Y} p(x, y) \log(p(x, y)) &&
\end{flalign*}

Нека имаме още множество  от характеристични функции $\mathcal{H}, |\mathcal{H}| = K$, които са от вида $h_i:X\times Y \rightarrow [0, 1]$.

Ако с $E(q, h)$ означим очакването на $h$, спрямо разпределение $q$, тоест:
\begin{flalign*}
	& E(q, h) = \sum\limits_{(x, y) \in X\times Y} q(x, y)h(x, y) &&
\end{flalign*}

То искаме за търсеното $p$ да е изпълнено:
\begin{flalign*}
	& E(p, h) = E(\tilde{p}, h), \forall h \in \mathcal{H} &&
\end{flalign*}
тоест
\begin{flalign*}
	E(p, h) & = \sum\limits_{(x, y) \in X\times Y} p(x, y)h(x, y) && \\
	& = \sum\limits_{(x, y)\in X\times Y} \tilde{p}(x)p(y|x)h(x, y) && \\
	& = \sum\limits_{(x, y) \in X\times Y} \tilde{p}(x, y)h(x, y) = E(\tilde{p}, h)
\end{flalign*}

Ако означим:
\begin{flalign*}
	& P = \{ p | E(p, h) = E(\tilde{p}, h), \forall h \in \mathcal{H} \}&&,
\end{flalign*}
тогава искаме да намерим 
\begin{flalign*}
	\hat{p} & = argmax_{p \in P} H(X, Y) &&\\
	& = argmax_{p \in P}\B{  - \sum\limits_{(x, y) \in X\times Y} p(x, y) \log(p(x, y))}&&\\
	& = argmax_{p \in P}\B{ -\sum\limits_{(x, y) \in X\times Y} p(x, y)\log(\tilde{p}(x)p(y|x)) }&&\\
	& = argmax_{p \in P}\B{ -\sum\limits_{(x, y) \in X\times Y} p(x, y)log(\tilde{p}(x))-\sum\limits_{(x, y) \in X\times Y} p(x, y)\log(p(y|x)) }&&\\
	& = argmax_{p \in P}\B{ -\sum\limits_{(x, y) \in X\times Y} \tilde{p}(x) p(y | x) log(\tilde{p}(x)) + H(Y|X) }&&\\
	& = argmax_{p \in P}\B{ -\sum\limits_{x\in X} \tilde{p}(x) log(\tilde{p}(x))\sum\limits_{y\in Y}p(y|x) + H(Y|X) }&&\\
	& = argmax_{p \in P}\B{ -\sum\limits_{x\in X} \tilde{p}(x) log(\tilde{p}(x)) + H(Y|X) }&&\\
	&  -\sum\limits_{x\in X} \tilde{p}(x)log(\tilde{p}(x)) \text{ е константа спрямо } p \text{, следователно:} &&\\
	& = argmax_{p\in P} \B{H(Y|X)}
\end{flalign*}

За да решим тази оптимизационна задача, ще ползваме множители на Лагранж. Тъй като имаме $K$ ограничения за всяка от характеристичните функции и трябва да отчетем, че търсим разпределение, задачата ще има вида:
\[
\Lambda(p, \lambda) = H(Y|X) + \sum\limits_{i=1}^{K} \lambda_i (E(p, h_i) - E(\tilde{p}, h_i )) + \sum\limits_{x\in X} \mu_x \sum\limits_{y \in Y} p(y|x) -1
\]

Нека фиксираме едно $x_0 \in X, y_0 \in Y$.
\begin{flalign*}
	& \cfrac{\partial\B{\Lambda(p, \lambda)}}{\partial p(y_0 | x_0)} && = \\
	& = \cfrac{\partial H(Y|X)}{\partial p(y_0 | x_0)} + \cfrac{\partial\B{\lambda_i (E(p, h_i) - E(\tilde{p}, h_i ))}}{\partial p(y_0 | x_0)} + \cfrac{\B{\sum\limits_{x\in X} \mu_x \sum\limits_{y \in Y} p(y|x) -1}}{\partial p(y_0 | x_0)}&&\\
	& = \cfrac{\partial\B{-\sum\limits_{(x, y) \in X\times Y} \tilde{p}(x)p(y | x)\log(p(y|x))}}{\partial p(y_0 | x_0)} + \cfrac{\partial\B{\sum\limits_{i=1}^K \lambda_i\Q{\sum\limits_{(x, y)\in X\times Y} \tilde{p}(x) h_i(x, y)\B{p(y|x) - \tilde{p}(y|x)} }}}{\partial p(y_0 | x_0)} + \mu_{x_0}&& \\
	\\
	& = -\tilde{p}(x_0) log(p(y_0 | x_0)) - \tilde{p}(x_0) + \sum\limits_{i=1}^K \lambda_i \tilde{p}(x_0) h_i(x_0, y_0) + \mu_{x_0}
\end{flalign*}

Искаме да я нулираме:

\begin{flalign}
	\label{appendix:max_ent:00}
	\nonumber &-\tilde{p}(x_0) log(p(y_0 | x_0)) - \tilde{p}(x_0) + \sum\limits_{i=1}^K \lambda_i \tilde{p}(x_0) h_i(x_0, y_0) + \mu_{x_0}= 0 \longleftrightarrow &&\\
	\nonumber &\tilde{p}(x_0) log(p(y_0 | x_0)) = \sum\limits_{i=1}^K \lambda_i \tilde{p}(x_0) h_i(x_0, y_0) - \tilde{p}(x_0) + \mu_{x_0}\longleftrightarrow && \\
	\nonumber & ln(p(y_0 | x_0)) = \sum\limits_{i=1}^K \lambda_i h_i(x_0, y_0) - 1 + \cfrac{\mu_{x_0}}{\tilde{p}(x_0)} \longleftrightarrow&&\\
	& p(y_0 | x_0) = \exp\B{\sum\limits_{i=1}^K \lambda_i h_i(x_0, y_0) - 1 + \cfrac{\mu_{x_0}}{\tilde{p}(x_0)}}
\end{flalign}

Производната по $\mu_{x_0}$ ни дава:
\begin{flalign*}
	&\sum\limits_{y\in Y} p(y | x_0) = 1 \longleftrightarrow && \\
	& \sum\limits_{y\in Y} \exp\B{\sum\limits_{i=1}^K \lambda_i h_i(x_0, y_0) - 1 + \cfrac{\mu_{x_0}}{\tilde{p}(x_0)}} = 1 \longleftrightarrow && \\
	& exp\B{-1 + \cfrac{\mu_{x_0}}{\tilde{p}(x_0)}}\sum\limits_{y\in Y} \exp\B{\sum\limits_{i=1}^K \lambda_i h_i(x_0, y_0)} = 1 &&\\
	& \longleftrightarrow &&\\
	& exp\B{-1 + \cfrac{\mu_{x_0}}{\tilde{p}(x_0)}} = \cfrac{1}{\sum\limits_{y\in Y} \exp\B{\sum\limits_{i=1}^K \lambda_i h_i(x_0, y_0)}} &&\\
	& \text{Заместваме в \autoref{appendix:max_ent:00}:} &&\\
	& p(y_0 | x_0) = \cfrac{\exp\B{\sum\limits_{i=1}^K \lambda_i h_i(x_0, y_0)}}{\sum\limits_{y\in Y} \exp\B{\sum\limits_{i=1}^K \lambda_i h_i(x_0, y_0)}} &&
\end{flalign*}

Следователно вида на търсеното $\hat{p}$ е $\hat{p}(x, y) = \pi\prod\limits_{i=1}^{K} e^{\lambda_i h_i(x, y)}$, като $\pi$ е нормализиращата константа. Ще покажем че $\hat{p}$, което минимизира ентропията също минимизира и условното правдоподобие.

Нека с $Q$ означим всички разпределения с желание вид. Имаме, че $Q = \{p \ | \  p(x, y) = \pi\prod\limits_{i = 1}^{K} e^{\lambda_i h_i(x, y)}\}$. За да намерим оптималното разпределение, ще ни е нужно да дефинираме разстояние между разпределения - ``Разстояние'' на Кулбек-Лайблър:
\begin{flalign*}
	D(p, q) = \sum\limits_{(x, y) \in X\times Y} p(x, y) \log\B{\cfrac{p(x, y)}{q(x, y)}}
\end{flalign*}

``Разстоянието'' на Кулбек-Лайблър всъщност не е разстояние в математическия смисъл (в смисъла на метрика), тъй като не е симетрична функция, но често се използва за разстояние между разпределения, тъй като има този интуитивен смисъл. Затова ще продължим да го наричаме разстояние, пропускайки кавичките.

С това сме готови да покажем следните твърдения:

\begin{lemma}
	\label{appendix:max_ent:01}
	За всеки две разпределения $p$ и $q$, $D(p, q) \geq 0$, като $D(p, q) = 0 \iff p = q$

	\begin{proof}
		Тъй като $p$ е разпределение и е изпълнено, че $\sum\limits_{(x, y) \in X\times Y} p(x, y) = 1$, можем да приложим неравенството на Йенсен:
		\begin{flalign*}
			& \sum\limits_{i = 1}^n p(x_i, y_i) f(z_i) \leq f\B{\sum\limits_{i = 1}^{n} p(x_i, y_i) z_i}, \forall (z_1,\ldots, z_n) \in \mathbb{R}^n, &&
		\end{flalign*}
		където $f$ е вдлъбната. Ако $f$ е строго вдлъбната ($f'$ е строго намаляваща), равенство се достига, когато $z_i$ е константа.

		\begin{flalign*}
			-D(p, q) & = -\sum\limits_{(x, y) \in X\times Y} p(x, y) \log\B{\cfrac{p(x, y)}{q(x, y)}}&&\\
			& = \sum\limits_{(x, y) \in X\times Y} p(x, y) \log\B{\cfrac{q(x, y)}{p(x, y)}}&&\\
			& \leq \log\B{\sum\limits_{(x, y) \in X\times Y} \cancel{p(x, y)}\cfrac{q(x, y)}{\cancel{p(x, y)}} }&&\\
			& \leq \log\B{\sum\limits_{(x, y) \in X\times Y} q(x, y)} = 0 &&\\
			& \iff D(p, q) \geq 0
		\end{flalign*}

		Тъй като логаритъмът е строго вдлъбната функция, равенство при неравенството на Йенсен се достига, когато $\cfrac{q(x, y)}{p(x, y)}$ е константа, тоест $\cfrac{q(x, y)}{p(x, y)} = 1 \iff p(x, y) = q(x, y)$ за произволно $(x, y) \in X\times Y$.
	\end{proof}
\end{lemma}

\begin{lemma}
	\label{appendix:max_ent:02}
	За всеки $p_1, p_2 \in P, q\in Q$ е изпълнено $\sum\limits_{(x, y) \in X\times Y} p_1(x, y) \log(q(x, y)) = \sum\limits_{(x, y) \in X\times Y} p_2(x, y) \log(q(x, y))$

	\begin{proof}
		\begin{flalign*}
			&\sum\limits_{(x, y) \in X\times Y} p_1(x, y) \log(q(x, y))&&\\
			& = \sum\limits_{(x, y) \in X\times Y} p_1(x, y)\log\B{\pi\prod\limits_{h_i \in \mathcal{H}} e^{\lambda_i h_i(x, y)}} &&\\
			& = \sum\limits_{(x, y) \in X\times Y} p_1(x, y)\B{\log(\pi) + \log\B{ \prod\limits_{h_i \in \mathcal{H}} e^{\lambda_i h_i(x, y)}}} &&\\
			& = \sum\limits_{(x, y) \in X\times Y} p_1(x, y)\B{\log(\pi) + \sum\limits_{h_i \in \mathcal{H}} \log\B{ e^{\lambda_i h_i(x, y)}}} &&\\
			& = \sum\limits_{(x, y) \in X\times Y} p_1(x, y)\B{\log(\pi) + \sum\limits_{h_i \in \mathcal{H}} \lambda_i h_i(x, y)} &&\\
			& = \sum\limits_{(x, y) \in X\times Y} p_1(x, y)\log(\pi) + \sum\limits_{(x, y) \in X\times Y} p_1(x, y)\sum\limits_{h_i \in \mathcal{H}} \lambda_i h_i(x, y) &&\\
			& = \log(\pi)\sum\limits_{(x, y) \in X\times Y} p_1(x, y) + \sum\limits_{(x, y) \in X\times Y} \sum\limits_{h_i \in \mathcal{H}} p_1(x, y) \lambda_i h_i(x, y) &&\\
			& = \log(\pi). 1 + \sum\limits_{(x, y) \in X\times Y} \sum\limits_{h_i \in \mathcal{H}} p_1(x, y) \lambda_i h_i(x, y) &&\\
			& = \log(\pi). 1 +  \sum\limits_{h_i \in \mathcal{H}} \lambda_i \sum\limits_{(x, y) \in X\times Y} p_1(x, y)  h_i(x, y) &&\\
			&\text{Тъй като } p_2 \in P: &&\\
			& = \log(\pi). 1 +  \sum\limits_{h_i \in \mathcal{H}} \lambda_i \sum\limits_{(x, y) \in X\times Y} p_2(x, y)  h_i(x, y) &&\\
			& \text{Използваме и че } \sum\limits_{(x, y) \in X\times Y} p_2(x, y) = 1 &&\\
			& = \log(\pi)\sum\limits_{(x, y) \in X\times Y} p_2(x, y) +  \sum\limits_{h_i \in \mathcal{H}} \lambda_i \sum\limits_{(x, y) \in X\times Y} p_2(x, y)  h_i(x, y) &&\\
			&\sum\limits_{(x, y) \in X\times Y} p_2(x, y) \log(q(x, y))&&\\
		\end{flalign*}
	\end{proof}
\end{lemma}

\begin{lemma}
	\label{appendix:max_ent:03}
	Ако $p \in P, q \in Q, r \in P\cap Q$, то $D(p, q) = D(p, r) + D(r, q)$

	\begin{proof}
		\begin{flalign*}
			& D(p, r) + D(r, q) &&\\
			& = \sum\limits_{(x, y)\in X\times Y}p(x, y)\log\B{\cfrac{p(x, y)}{r(x, y)}} + \sum\limits_{(x, y)\in X\times Y}r(x, y)\log\B{\cfrac{r(x, y)}{q(x, y)}}&&\\
			& = \sum\limits_{(x, y)\in X\times Y}p(x, y)\log(p(x, y)) - \sum\limits_{(x, y)\in X\times Y}p(x, y) \log(r(x, y)) + &&\\
			& \quad\sum\limits_{(x, y)\in X\times Y}r(x, y)\log(r(x, y)) - \sum\limits_{(x, y)\in X\times Y}r(x, y) \log(q(x, y))&&\\
			& \coolleq{\autoref{appendix:max_ent:01}}
			& &&\\
			&\quad \sum\limits_{(x, y)\in X\times Y}p(x, y)\log(p(x, y)) - \cancel{\sum\limits_{(x, y)\in X\times Y}p(x, y) \log(r(x, y))} + &&\\
			& \cancel{\quad\sum\limits_{(x, y)\in X\times Y}{\color{red}p(x, y)}\log(r(x, y))} - \sum\limits_{(x, y)\in X\times Y}r(x, y) \log(q(x, y))&&\\
			& &&\\
			& = \sum\limits_{(x, y)\in X\times Y}p(x, y)\log(p(x, y)) - \sum\limits_{(x, y)\in X\times Y}r(x, y) \log(q(x, y)) &&\\
			& \coolleq{\autoref{appendix:max_ent:01}} &&\\
			& \quad \sum\limits_{(x, y)\in X\times Y}p(x, y)\log(p(x, y)) - \sum\limits_{(x, y)\in X\times Y}{\color{red} p(x, y)} \log(q(x, y)) &&\\
			& = \sum\limits_{(x, y)\in X\times Y}p(x, y) \log\B{\cfrac{p(x, y)}{q(x, y)}} = D(p, q) &&\\
		\end{flalign*}
	\end{proof}
\end{lemma}

\begin{lemma}
	\label{appendix:max_ent:04}
	Ако $r \in P\cap Q$, то $r$ е единствено и $r = \hat{p}$

	\begin{proof}

		Нека $r \in P\cap Q$. Условието $r = \hat{p}$ значи, че $r = argmax_{p \in P} H(p)$, тоест ще покажем, че за всяко $p \in P: H(r) \geq H(p)$.

		Нека $u$ е равномерното разпределение върху $X\times Y$, тоест $u(x, y) = \cfrac{1}{n}, \forall (x, y)\in X\times Y$. Следователно $u \in Q$, защото можем да изберем $\pi = \cfrac{1}{n}$ и $\lambda_i = 0, \forall i\in \{1\ldots K\}$

		Нека фиксираме произволно $p \in P$. Тогава от \autoref{appendix:max_ent:03} следва, че
		\begin{flalign*}
			& D(p, u)  = D(p, r) + D(r, u) &&\\
			& D(p, u) \quad \coolgeq{\autoref{appendix:max_ent:01}} \quad D(r, u) &&\\
			& \sum\limits_{(x, y) \in X\times Y} p(x, y) \log\B{\cfrac{p(x, y)}{u(x, y)}} \geq \sum\limits_{(x, y) \in X\times Y} r(x, y) \log\B{\cfrac{r(x, y)}{u(x, y)}} && \\
			& -H(p) - \sum\limits_{(x, y) \in X\times Y} p(x, y) \log(u(x, y)) \geq -H(r) - \sum\limits_{(x, y) \in X\times Y} r(x, y) \log(u(x, y)) &&\\
			& \coollra{\autoref{appendix:max_ent:02}} &&\\
			& -H(p) - \cancel{\sum\limits_{(x, y) \in X\times Y} p(x, y) \log(u(x, y))} \geq -H(r) - \cancel{\sum\limits_{(x, y) \in X\times Y} {\color{red} p(x, y)} \log(u(x, y))} &&\\
			& H(r) \geq H(p) &&
		\end{flalign*}
		Следователно $r = argmax_{p \in P} H(p)$

		Сега нека видим защо $r$ е единствено.
		Нека $r' = argmax_{p \in P} H(P)$. Тогава:
		\begin{flalign*}
			& H(r') = H(r) \longleftrightarrow D(r, u) = D(r', u) && \\
			& \text{ но } D(r, u) = D(r, r') + D(r', u) \text{ по }\autoref{appendix:max_ent:02} && \\
			& \Longrightarrow \quad D(r, r') = 0 &&\\
			& \coolra{\autoref{appendix:max_ent:01}} \quad r = r' &&\\
		\end{flalign*}
	\end{proof}
\end{lemma}

Сега, нека дефинираме правдоподобие на разпределение $p$ като при дадено множество $\mathcal{D}$:
\begin{flalign*}
	& \widehat{L}_{\mathcal{D}}(p) = \prod\limits_{(x, y) \in X\times Y} p(x, y) &&
\end{flalign*}
Тъй като логаритъмът е вдлъбната и монотонно растяща функция, често се разглежда за удобство:
\begin{flalign*}
	& log\B{\widehat{L}_{X\times Y}(p)} = \sum\limits_{(x, y) \in \mathcal{D}} log(p(x, y)) &&
\end{flalign*}

Дефинираме фунцкията $L(p)$:
\begin{flalign*}
	L(p) = \sum\limits_{(x, y) \in X\times Y} \tilde{p}(x, y) \log(p(x, y)) & = \sum\limits_{(x, y) \in D} \tilde{p} (x, y) \log(p(x, y)) + \sum\limits_{(x, y) \notin D} \tilde{p}(x, y) \log(p(x, y)) && \\
	& = \sum\limits_{(x, y) \in D} \tilde{p} (x, y) \log(p(x, y)) + 0 && \\
	& = \sum\limits_{(x, y) \in D} \cfrac{1}{n} \log(p(x, y)) = C\log\B{\widehat{L}_{X\times Y}(p)} && \\
\end{flalign*}
Тоест $L$ е пропорционална на логаритъм от правдоподобието. Тоест, ако $p$ максимизира $L$, то максимизира и правдоодобието, и обратното.

\begin{lemma}
	\label{appendix:max_ent:05}
	Ако $r \in P\cap Q$, то $r$ е единствено и $r = argmax_{q \in Q} L(q)$

	\begin{proof}
		Искаме да покажем, че за всяко $q \in Q: L(r) \geq L(q)$.

		Нека фиксираме едно $q \in Q$, а $\tilde{p}$ е емпиричното разпределение и следователно $\tilde{p} \in P$.

		Тогава от \autoref{appendix:max_ent:03} следва, че:
		\begin{flalign*}
			& D(\tilde{p}, q)  = D(\tilde{p}, r) + D(r, q) &&\\
			& D(\tilde{p}, q) \quad \coolgeq{\autoref{appendix:max_ent:01}} \quad D(\tilde{p}, r) &&\\
			& \sum\limits_{(x, y) \in X\times Y} \tilde{p}(x, y) log\B{\cfrac{\tilde{p}(x, y)}{q(x,y)}} \geq \sum\limits_{(x, y) \in X\times Y} \tilde{p}(x, y) log\B{\cfrac{\tilde{p}(x, y)}{r(x, y)}} &&\\
			& -\cancel{H(\tilde{p})} - L(q) \geq -\cancel{H(\tilde{p})} - L(r) && \\
			& \longleftrightarrow L(r) \geq L(q)
		\end{flalign*}

		Сега нека $r' = argmax_{q \in Q} L(q)$, тоест $L(r) = L(r') \Longrightarrow D(\tilde{p}, r) = D(\tilde{p}, r')$.

		Но по \autoref{appendix:max_ent:03}, $D(\tilde{p}, r') = D(\tilde{p}, r) + D(r, r') \Longrightarrow D(r, r') = 0 \coollra{\autoref{appendix:max_ent:01}} r' = r$, следователно $r$ е единствено.
	\end{proof}
\end{lemma}

От \autoref{appendix:max_ent:04} и \autoref{appendix:max_ent:05}, че ако вземем разпределение от сечението на $P$ и $Q$, то е единствено и е равно на $\hat{p} = argmax_{p \in P} H(p) = argmax_{q \in Q} L(q)$. Тъй като $L$ е пропорционално на правдоподобието, за да намерим търсеното разпределение е достатъчно да максимизираме правдоподобието.
\end{document}
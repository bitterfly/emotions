\documentclass[main.tex]{subfiles}

\begin{document}
\appendix
\chapter{Фурие приложение}
\label{appendix:fourier}
    \section{Дефиниция}

    Понякога е по-лесно да се моделира поведението на система, ако можем да кажем как ще се държи системата за
    всяка честота поотделно. Например, по този начин можем да нулираме всички честоти под или над дадена
    или да усилим определена честоти.
    За тази цел ни трябва еквивалентно представяне на даден сигнал във времето като съвкупност от синусоиди с различни честоти.
    Нека имаме дискретен във времето сигнал $x$, който е периодичен с фундаментален период T, измерен в секунди. Тоест, 
    \begin{flalign*}
        & x(t) = x(t + T) &&
    \end{flalign*}

    Честотата, изразена в херци (периоди в секунда), се означава с $f_0 = \frac{1}{T}$ и означава "брой периоди в секунда". Нарича се фундаментална честота.
    Честотата, изразена в радиани в секунда, се означава с $\omega_0 = f_0 2\pi$ и се нарича фундаментална ъглова честота.

    Тогава представянето, което търсим има вида:

    \begin{flalign}
        \label{eq:appendix:0:1}
        & x(t) = \sum\limits_{k=-\infty}^{\infty} a_k e^{\frac{2k\pi i t}{T}}, &&
    \end{flalign}
    
    където $e^{\frac{2k\pi i t}{T}}$ е сигнал с честота $\cfrac{k}{T}$
    Представянето от $\autoref{eq:appendix:0:1}$ се нарича развиване в ред на Фурие за сигнала $x(t)$
    
    Нека намерим вида на коефициентите $a_k$

    Умножаваме \autoref{eq:appendix:0:1} с $e^{-\frac{2n\pi i t}{T}}$, тоест
    \begin{flalign*}
        & x(t)e^{-\frac{2n\pi i t}{T}} = \sum\limits_{k=-\infty}^{\infty} a_k e^{\frac{2k\pi i t}{T}} e^{-\frac{2n\pi i t}{T}} &&
    \end{flalign*}
    Ако интегрираме двете страни от 0 до фундаменталния период T, получаваме

    \begin{flalign*}
        & \int\limits_{0}^{T} x(t)e^{-\frac{2n\pi i t}{T}} = \int\limits_{0}^{T} \sum\limits_{k=-\infty}^{\infty} a_k e^{\frac{2k\pi i t}{T}} e^{-\frac{2n\pi i t}{T}} &&\\
        & \int\limits_{0}^{T} x(t)e^{-\frac{2n\pi i t}{T}} =  \sum\limits_{k=-\infty}^{\infty} a_k \Q{\int\limits_{0}^{T} e^{\frac{2(k - n)\pi i t}{T}}} &&
    \end{flalign*}

    Да разгледаме $\int\limits_{0}^{T} e^{\frac{2(k - n)\pi i t}{T}}$
    \begin{flalign*}
        & \int\limits_{0}^{T}e^{\frac{2(k - n)\pi i t}{T}} = \int\limits_{0}^{T} \cos(\frac{2(k - n)\pi t}{T}) dt + i \int\limits_{0}^{T} \sin(\frac{2(k - n)\pi t}{T}) dt = && \\
        & =  \begin{cases}
            1\Big|_0^T + 0, & n = k\\
            0 + 0, & \text{иначе}   
        \end{cases} && \\
        & = \begin{cases}
            T, & n = k \\
            0, & \text{иначе}
        \end{cases} && 
    \end{flalign*}

    Което означава, че

    \begin{flalign*}
        & a_n = \cfrac{1}{T} \int\limits_{0}^{T} x(t)e^{-\frac{2n\pi i t}{T}} &&
    \end{flalign*}
    Това е вярно и за всеки друг интервал с дължина T:

    \begin{flalign}
        \label{eq:appendix:0:2}
        & a_n = \cfrac{1}{T} \int\limits_{T} x(t)e^{-\frac{2n\pi i t}{T}} &&
    \end{flalign}

    Обикновено с $X(e^{i\omega})$ означаваме $\cfrac{1}{T} \int\limits_{T} x(t)e^{-i\omega}$,
    където $\omega$ е ъглова честота.

    Може да се покаже, че редът на Фурие за някакъв сигнал $x(t)$ е сходящ и съответно коефициентите от \autoref{eq:appendix:0:2} са крайни, ако е изпълнено, че

    \begin{flalign*}
        & \int\limits_{T} |x(t)|^2 < \infty, &&
    \end{flalign*}

    например в Oppenheim 96. 

    Още повече, ако сигналът $x$ е дискретен и периодичен (какъвто е случаят, когато семплираме речеви сигнал) и периодичен $x[n] = x[n + N]$,
    имаме само N различни стойности.

    $e^{\frac{2 (k + N) \pi i n}{N}} = e^{\frac{2 k \pi i n}{N}} e^{\frac{2 N \pi i n}{N}} = e^{\frac{2 k \pi i n}{N}}$, тъй като
    $e^{2\pi i n} = \cos(2\pi n) + \sin(2\pi n) = 1$
    
    следователно са ни достатъчни само кои да е N последователни стойности:
    \begin{flalign}
        \label{eq:appendix:0:3}
        & \nonumber x[n] =  \sum\limits_{k=-\infty}^{\infty} \hat{a_k} e^{\frac{2 k \pi i n}{N}} && \\
        & x[n] = \sum\limits_{k=0}^{N-1} a_k e^{\frac{2 k \pi i n}{N}} &&  
    \end{flalign}

    \autoref{eq:appendix:0:3} се нарича ред на Фурие за дискретен във времето сигнал.

    Коефициентите можем да намерим по същия начин като в непрекъснатия случай, но използвайки сума, вместо интеграл:

    \begin{flalign*}
        & \sum\limits_{n=0}^{N-1} x[n] e^{\frac{-2\pi i r n}{N}} = \sum\limits_{n=0}^{N-1} \sum\limits_{k=0}^{N-1} a_k e^{\frac{2\pi i k n}{N}} e^{\frac{-2\pi i r n}{N}}  = && \\
        & \sum\limits_{n=0}^{N-1} x[n] e^{\frac{-2\pi i r n}{N}} = \sum\limits_{k=0}^{N-1} a_k \sum\limits_{n=0}^{N-1} e^{\frac{2\pi i (k-r) n}{N}} &&
    \end{flalign*}

    и отново използваме, че 

    \begin{flalign}
        \label{eq:appendix:0:4}
        & \nonumber \sum\limits_{n=0}^{N-1} e^{\frac{2\pi i (k-r) n}{N}} = \begin{cases}
            N, & k-r \equiv 0 (mod N) \\
            0, & \text{иначе}
        \end{cases} && \\
        & \Rightarrow a_r = \cfrac{1}{N} \sum\limits_{n=0}^{N-1} x[n] e^{-\frac{2\pi i r n}{N}} &&
    \end{flalign}

    което е изпълнено и за всеки друг интервал с дължина N.
    Ще използваме означенията $x(t) \xleftrightarrow{\mathcal{F}\mathcal{S}} a_k$
    или $x(t) \xleftrightarrow{\mathcal{F}\mathcal{S}} X(e^{i\omega_k})$,
    
    където
    $a_k = X(e^{\frac{2\pi i k}{N}}) = X(e^{i\omega_k})$ за $\omega_k = \frac{2\pi k}{N}$


    \section{Свойства}

    \begin{itemize}
        \item Изместване във времето

        Ако $x[n] \xleftrightarrow{\mathcal{F}\mathcal{S}} a_k$, то $x[n-n_0] \xleftrightarrow{\mathcal{F}\mathcal{S}} b_k = a_k e^{-\frac{2\pi i n_0}{N}}$
       
        Тъй като \autoref{eq:appendix:0:4} е изпълнено за всеки интервал, то можем да изберем интервала $[n_0, T-1+n_0]$
        \begin{flalign*}
            & b_k = \cfrac{1}{N} \sum\limits_{n = n_0}^{N-1 + n_0} x[n - n_0] e^{-\frac{2\pi i k n}{N}} = \sum\limits_{n=n_0}^{N-1+n_0} x[n - n_0] e^{-\frac{2\pi i k (n - n_0)}{N}} e^{-\frac{2\pi i k n_0}{N}} = &&\\
            & e^{-\frac{2\pi i k n_0}{N}} \sum\limits_{\tau=0}^{T} x[\tau] e^{-\frac{2\pi i k \tau}{N}} = e^{-\frac{2\pi i k n_0}{N}} a_k&&
        \end{flalign*}
        \item Симетричност на комплексно спрегнатите за реален сигнал
        
        Ако $x[n] = \bar{x}[n]$ е реален сигнал, за който $x(t) \xleftrightarrow{\mathcal{F}\mathcal{S}} a_k$, то $\overline{a}_k = a_{-k}$
        
        От уравнение \autoref{eq:appendix:0:4} следва, че:
        \begin{flalign*}
            & a_k = \cfrac{1}{N} \sum\limits_{n=0}^{N-1} x[n] e^{-\frac{2\pi i k n}{N}} && \\
            & \overline{a_k} = \cfrac{1}{N} \sum\limits_{n=0}^{N-1} \overline{x[n] e^{-\frac{2\pi i k n}{N}}} = \sum\limits_{n=0}^{N-1} \overline{x[n]} e^{\frac{2\pi i k n}{N}} =  \sum\limits_{n=0}^{N-1}  x[n] e^{\frac{2\pi i k n}{N}} && \\
            & \Rightarrow \overline{a_k} = a_{-k}
        \end{flalign*}
    \end{itemize}

    \begin{theorem}[Теорема за конволюцията]
    \label{th:appendix:fourier:convolution}
    \end{theorem}
\end{document}
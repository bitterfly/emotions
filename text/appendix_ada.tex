\documentclass[main.tex]{subfiles}
\begin{document}
\chapter{Приложение за AdaBoost}
\label{appendix:ada}
Ще разгледаме алгоритъма AdaBoost в дискретния случай, като доказателството следва това в samme.

Задачата е следната:
Дадено е $K$-брой класове. Имаме тренировъчни данни $M := (x_1, y_1), (x_2, y_2), \cdots (x_n, y_n), x_i \in X, y_i \in \{1\ldots K\}, i = 1 \ldots n$ и множество $\mathcal{H}$ от класификатори от вида $h_i : X\times \{1\ldots K\} \rightarrow \{0, 1\}$, където 
\begin{flalign*}
	& h_i(x, y) = \begin{cases} 
		1, & \text{ако класът на } $x$ \text{ е } $y$\\
		0, & \text{иначе} \\
	\end{cases} &&
\end{flalign*}	
и съответно $\sum\limits_{j=1}^K h_i(x, j) = 1$. При дадена константа $T$, търсим линейна комбинация на функции от $\mathcal{H}$:
\[
	H = \sum\limits_{i=1}^T \alpha_i h_i
\]
която минимизира загубата на точност.
По-точно, искаме да оптимизираме следната функция на загубата:
\begin{flalign*}
	& L_{0, 1}(g, x, y) = \begin{cases} 
		0, & g(x, y) = max(g(x, 1), \ldots g(x, K))\\
		1, & \text{иначе} \\
	\end{cases} &&
\end{flalign*}
Тоест, имаме загуба 1, ако класификаторът ни не е познал правилно класа и 0, в противен случай.
Това може да се запише и по следния начин:
\begin{flalign*}
	& L_{0, 1}(g, x, y) = \begin{cases} 
		0, & y = argmax_j(g(x, j))\\
		1, & \text{иначе} \\
	\end{cases} &&
\end{flalign*}
съответно можем да добавим константа, която няма да промени минимизацията:

\begin{flalign*}
	& L_{0, 1}(g, x, y) = \begin{cases} 
		0, & y = argmax_j\B{g(x, j) - \cfrac{1}{K}\B{\sum\limits_{k=1}^K g(x, k)}}\\
		1, & \text{иначе} \\
	\end{cases} = \begin{cases} 
		0, & g(x, y) - \cfrac{1}{K}\B{\sum\limits_{j=1}^K g(x, j)} > 0\\
		1, & \text{иначе} \\
	\end{cases} &&
\end{flalign*}

Нека означим $\tilde{g}(x, y) = g(x, y) - \cfrac{1}{K}\B{\sum\limits_{j=1}^K g(x, j)}$. Тоест, искаме да оптимизираме:
\begin{flalign*}
	& L_{0, 1}(g, x, y) = \begin{cases} 
		0, & \tilde{g}(x, y) > 0\\
		1, & \text{иначе} \\
	\end{cases} &&
\end{flalign*}

Тъй като тази функция не е диференцируема, можем за удобство да използваме експоненциалната функция на загубата, дефинирана по следния начин:
\[
\mathcal{l}(g, x, y) = e^{-\tilde{g}(x, y)}
\]

Освен че тя е диференцируема, имаме, че ако $\tilde{g}(x, y) > 0, l(g, x, y) > 0$ и ако $\tilde{g}(x, y) \leq 0, l(g, x, y) \geq 1$, което означава, че $l(g, x, y)$ ограничава отгоре $L_{0, 1}(g, x, y)$ и в някакъв смисъл е ``по-песимистична''.

AdaBoost алгоритъмът избира последователно функции $h_i$ от $\mathcal{H}$ и намира за всяка тегло $\alpha_i$.

С $H_t$ ще означаваме линейната комбинация, получена от първите $t$ избрани класификатори. Имаме, че:
\begin{flalign}
	\label{appendix:ada:01}
	\nonumber H_t(x, y) & = \sum\limits_{i=1}^t \alpha_i h_i(x, y)\\
	\nonumber & = H_{t-1}(x, y) + \alpha_t h_t(x, y)\\
	\nonumber\\
	\nonumber \tilde{H}_{t}(x, y) & = && \\
	\nonumber& = H_{t-1}(x, y) + \alpha h_t(x, y) - \cfrac{1}{K}\B{\sum\limits_{j=1}^K H_{t-1}(x, j) + \alpha h_t(x, j)} &&\\
	\nonumber& = H_{t-1}(x, y) - \cfrac{1}{K}\B{\sum\limits_{j=1}^K H_{t-1}(x, j)} + \alpha\Q{h_{t}(x, y) - \cfrac{1}{K}\B{\sum\limits_{j=1}^K h_{t}(x, y)}} &&\\
	& = \tilde{H}_{t-1}(x, y) + \alpha \tilde{h}(x, y) &&\\
	\label{appendix:ada:02}
	H_0(x, y) & = 0 \text{ и } \tilde{H}_0(x, y) = 0\quad \forall x \in X, \forall y \in \{1\ldots K\} &&
\end{flalign}

На всяка итерация $t$ на алгоритъма, дефинираме разпределение върху тренировъчните данни, което ще означаваме с $\mathcal{D}_t$, където $\mathcal{D}_t(i)$ дава вероятност на $i$-тия пример. Идеята е да може да се даде по-голяма вероятност на тези примери, върху които предните $t-1$ избрани функции бъркат и да се избере тази функция, която се представя най-добре върху така претеглените данни. В началото на алгоритъма за $\mathcal{D}_1$ избираме равномерно разпределение, тоест:
\[\mathcal{D}_1(i) = \cfrac{1}{n}, i = 1\cdots n\]

Алгоритъмът е следният:

\begin{exampleenv}
	\begin{alg}
		\item $D_1(i) = \cfrac{1}{n}, i = 1\cdots n$
		\item $H_0 = \emptyset$
		\item За всяко $t$ от 1 до $T$ се прави следното:
		\begin{alg}
			\item $h_t = argmin_{h\in \mathcal{H}} P_{i\sim D_t}\B{\tilde{h}(x_i, y_i) \leq 0}$
			\item $\varepsilon_t = P_{i\sim D_t}\B{\tilde{h}(x_i, y_i) \leq 0}$
			\item $\alpha_t = \ln\B{\cfrac{1 - \varepsilon_t}{\varepsilon_t}} + ln\B{K-1}$
			\item $H_t = H_{t-1} \cup \{(h_t, \alpha_t)\}$
			\item За всяко i от 1 до n:
			\begin{alg}
				\item $\mathcal{D}_{t+1}(i) = \cfrac{\mathcal{D}_t(i) e^{-\alpha_t \tilde{h}_t(x_i, y_i)}}{\sum\limits_{j=1}^n \mathcal{D}_t(j) e^{-\alpha_t \tilde{h}_t(x_i, y_j)}}$
			\end{alg} 
		\end{alg}
		\item $H = H_T$
		\item Връщаме $H$
	\end{alg}
\end{exampleenv}

Целта на AdaBoost алгоритъмът е да се намери такова $H$, което минимизира експоненциалната грешка $\mathcal{l}$.

За да видим, че горният алгоритъм намира такова $H$, ще докажем няколко твърдения. 

Първо нека положим $w_{t,i} = e^{-\tilde{H}_{t-1}(x_i, y_i)}$ за удобство.

\begin{lemma}
	$\mathcal{D}_t(i) = \cfrac{w_{t, i}}{\sum\limits_{j=1}^n w_{t, j}}$ 
\end{lemma}

\begin{proof}
	Ще го докажем по индукция.
	\begin{itemize}
		\item База $t = 1$
		\begin{flalign*}
			\cfrac{w_{1, i}}{\sum\limits_{j=1}^n w_{1, j}} & = \cfrac{e^{-\tilde{H}_{0}(x_i, y_i)}}{\sum\limits_{j=1}^n e^{\tilde{H}_{0}(x_j, y_j)}}&&\\
			& \stackrel{\mathclap{\normalfont\mbox{\tiny{\ref{appendix:ada:02}}}}}{=\joinrel=} \cfrac{1}{n} &&\\
			& \stackrel{\mathclap{\normalfont\mbox{\tiny{ред 1.}}}}{=\joinrel=} \mathcal{D}_1(i) \text{ за всяко i} &&
		\end{flalign*}
		\item Нека твърденито е изпълнено за $\mathcal{D}_t(i)$ за всяко $i$
		\item $\mathcal{D}_t(i) = \cfrac{w_{t, i}}{\sum\limits_{j=1}^n w_{t, j}} \Rightarrow \mathcal{D}_{t+1}(i) = \cfrac{w_{t+1, i}}{\sum\limits_{j=1}^n w_{t+1, j}}$
		\begin{flalign*}
			\mathcal{D}_{t+1}(i) & \coolleq{ред 3.5.1.} \cfrac{\mathcal{D}_t(i) e^{-\alpha_t \tilde{h}_t(x_i, y_i)}}{\sum\limits_{j=1}^n \mathcal{D}_t(j) e^{-\alpha_t \tilde{h}_t(x_j, y_j)} } \cooleq{ИХ} \cfrac{\cfrac{w_{t, i}}{\sum\limits_{k=1}^n w_{t, k}} e^{-\alpha_t \tilde{h}_t(x_i, y_i)}}{\sum\limits_{j=1}^n \cfrac{w_{t, j}}{\sum\limits_{k=1}^n w_{t, k}} e^{-\alpha_t \tilde{h}_t(x_j, y_j)} } && \\
			& \cooleq{def} \cfrac{\cfrac{e^{-\tilde{H}_{t-1}(x_i, y_i)}}{\sum\limits_{k=1}^n e^{-\tilde{H}_{t-1}(x_k, y_k)}} e^{-\alpha_t \tilde{h}_t(x_i, y_i)}}{\sum\limits_{j=1}^n \cfrac{e^{-\tilde{H}_{t-1}(x_j, y_j)}}{\sum\limits_{k=1}^n e^{-\tilde{H}_{t-1}(x_k, y_k)}} e^{-\alpha_t \tilde{h}_t(x_j, y_j)} } && \\
			& = \cfrac{e^{-(\tilde{H}_{t-1}(x_i, y_i) + \alpha_t \tilde{h}_t(x_i, y_i))}}{\cancel{\sum\limits_{k=1}^n e^{-\tilde{H}_{t-1}(x_k, y_k)}}} \cfrac{\cancel{\sum\limits_{k=1}^n e^{-\tilde{H}_{t-1}(x_k, y_k)}}}{\sum\limits_{j=1}^n e^{-(\tilde{H}_{t-1}(x_j, y_j) + \alpha_t \tilde{h}_t(x_j, y_j))}} && \\
			& = \cfrac{e^{-(\tilde{H}_{t-1}(x_i, y_i) + \alpha_t \tilde{h}_t(x_i, y_i))}}{\sum\limits_{j=1}^n e^{-(\tilde{H}_{t-1}(x_j, y_j) + \alpha_t \tilde{h}_t(x_j, y_j))}} \coolleq{\ref{appendix:ada:01}} \cfrac{e^{-\tilde{H}_t(x_i, y_i)}}{\sum\limits_{j=1}^n e^{-\tilde{H}_t(x_j, y_j) }} \cooleq{def} \cfrac{w_{t+1, i}}{\sum\limits_{j=1}^n w_{t+1, j}} && \\
		\end{flalign*}
		
	\end{itemize}
\end{proof}

\begin{lemma}
	\label{appendix:ada:lemma:1}
	$P_{i\sim D_t}\B{\tilde{h}(x_i, y_i) \leq 0} = \sum\limits_{i:\tilde{h}(x_i, y_i) \leq 0} \cfrac{w_{t, i}}{\sum\limits_{j=1}^n w_{t, j}}$
\end{lemma}

\begin{proof}
	Използвайки горното твърдение, имаме че:
	\begin{flalign*}
		P_{i\sim D_t}(\tilde{h}(x_i, y_i) \leq 0) = \sum \limits_{i:\tilde{h}(x_i, y_i) \leq 0} \mathcal{D}_t(x_i) = \sum\limits_{i:\tilde{h}(x_i, y_i) \leq 0} \cfrac{w_{t, i}}{\sum\limits_{j=1}^n w_{t, j}}
	\end{flalign*}
\end{proof}

\begin{lemma}
	\label{appendix:ada:lemma:2}
	Изборът на $h_t = argmin_{h\in \mathcal{H}} P_{i\sim D_t}(\tilde{h}(x_i, y_i) \leq 0)$ от точка 3.2 на алгоритъма минимизира експоненциалната грешка на $H_t$ върху тренировъчното множество, тоест:
	\[
		h_t = argmin_{h\in \mathcal{H}} \B{ \cfrac{1}{n}\sum\limits_{i=1}^n \mathcal{l}(H_{t-1} + Ch, x_i, y_i)},
		\]
		където $C$ е произволна константа.
\end{lemma}

\begin{proof}
\begin{flalign*}
	h_t & = argmin_{h\in \mathcal{H}} \B{ \cfrac{1}{n}\sum\limits_{i=1}^n \mathcal{l}(H_{t-1} + Ch, x_i, y_i)} &&\\
	& = argmin_{h\in \mathcal{H}} \B{ \cfrac{1}{n}\sum\limits_{i=1}^n e^{-(\tilde{H}_{t-1}(x_i, y_i) + C\tilde{h}(x_i, y_i))}} &&\\
	& = argmin_{h\in \mathcal{H}} \B{ \cfrac{1}{n}\sum\limits_{i=1}^n e^{-\tilde{H}_{t-1}(x_i, y_i)} e^{-C\tilde{h}(x_i, y_i)}} &&
\end{flalign*}
Махаме константата $\cfrac{1}{n}$, тъй като тя не влияе на минимизацията, и заместваме с $w_{t, i}$.

\begin{flalign*}
	& = argmin_{h\in \mathcal{H}} \B{\sum\limits_{i=1}^n w_{t, i} e^{-C\tilde{h}(x_i, y_i)}} &&\\
\end{flalign*}

Сега можем да разделим сумата на две, в зависимост $\tilde{h}$ е сбъркал, тоест дали $\tilde{h}(x_i, y_i) \leq 0$. Ако това е изпълнено, то $\tilde{h}(x_i, y_i) = h(x_i, y_i) - \cfrac{1}{K} \sum\limits_{j=1}^K\B{h(x_i, j)} = 0 - \cfrac{1}{K} = -\cfrac{1}{K}$ и е равно на $1 - \cfrac{1}{K}$ в противен случай, тоест:

\begin{flalign*}
	& = argmin_{h\in \mathcal{H}} \B{\sum\limits_{i: \tilde{h}(x_i, y_i) > 0}^n w_{t, i} e^{-C(1 - 1/K)} + \sum\limits_{i: \tilde{h}(x_i, y_i) \leq 0}^n w_{t, i} e^{C/K}} &&\\
	& = argmin_{h\in \mathcal{H}} \B{\Q{\sum\limits_{i=1}^n w_{t, i} e^{-C(1 - 1/K)} - \sum\limits_{i: \tilde{h}(x_i, y_i) \leq 0}^n w_{t, i} e^{-C(1 - 1/K)}}+ \sum\limits_{i: \tilde{h}(x_i, y_i) \leq 0}^n w_{t, i} e^{C/K}} &&\\
	& = argmin_{h\in \mathcal{H}} \B{\sum\limits_{i=1}^n w_{t, i} e^{-C(1 - 1/K)} + \sum\limits_{i: \tilde{h}(x_i, y_i) \leq 0}^n w_{t, i} (e^{C/K} - e^{-C(1 - 1/K)}) } &&\\
\end{flalign*}
$\sum\limits_{i=1}^n w_{t, i} e^{-C(1 - 1/K)}$ е константа спрямо $h$, затова също не участва в минимизацията и тогава:

\begin{flalign*}
	& = argmin_{h\in \mathcal{H}} \B{\sum\limits_{i: \tilde{h}(x_i, y_i) \leq 0}^n w_{t, i} (e^{C/K} - e^{-C(1 - 1/K)})} &&\\
	& = argmin_{h\in \mathcal{H}} \B{ (e^{C/K} - e^{-C(1 - 1/K)}) \sum\limits_{i: \tilde{h}(x_i, y_i) \leq 0}^n w_{t, i}} &&\\
\end{flalign*}
Нека $(e^{C/K} - e^{-C(1 - 1/K)}) > 0$. Този израз също не зависи от $h$: 
\begin{flalign*}
	& = argmin_{h\in \mathcal{H}} \B{\sum\limits_{i: \tilde{h}(x_i, y_i) \leq 0}^n w_{t, i}}, &&\\
\end{flalign*}
Можем да умножим по константата $\cfrac{1}{\sum\limits_{j=1}^n w_{t, j}}$ и да получим:
\begin{flalign*}
	& = argmin_{h\in \mathcal{H}} \B{\cfrac{\sum\limits_{i: \tilde{h}(x_i, y_i) \leq 0}^n w_{t, i}}{\sum\limits_{j=1}^n w_{t, j}} }, &&\\
	& = argmin_{h\in \mathcal{H}} \B{\sum\limits_{i: \tilde{h}(x_i, y_i) \leq 0}^n \cfrac{w_{t, i}}{\sum\limits_{j=1}^n w_{t, j}} }, &&\\
	& \coolleq{\autoref{appendix:ada:lemma:1}} \quad P_{i\sim D_t}(\tilde{h}(x_i, y_i) \leq 0) &&\\
\end{flalign*}
\end{proof}

\begin{lemma}
	\label{appendix:ada:lemma:3}
	Изборът на $\alpha_t$ от точка 3.3 на алгоритъма минимизира експоненциалната грешка на $H_t$ върху тренировъчното множество, тоест:
	\[
		\alpha_t = argmin_\alpha (\cfrac{1}{n} \sum\limits_{i=1}^n \mathcal{l}(H_{t-1} + \alpha h_t, x_i, y_i))
	\]
\end{lemma}

\begin{proof}

Имаме, че $\varepsilon_t = P_{i\sim D_t}(\tilde{h}(x_i, y_i) \leq 0) \quad \coolleq{\autoref{appendix:ada:lemma:1}} \quad \sum\limits_{i:\tilde{h}(x_i, y_i) \leq 0} \cfrac{w_{t, i}}{\sum\limits_{j=1}^n w_{t, j}}$

Ако заместим $C$ с $\alpha$ в предишното твърдение можем да получим, че:
\begin{flalign*}
	argmin_\alpha (\cfrac{1}{n} \sum\limits_{i=1}^n \mathcal{l}(H_{t-1} + \alpha h_t, x_i, y_i)) = argmin_\alpha \B{\sum\limits_{i: \tilde{h}(x_i, y_i) > 0}^n w_{t, i} e^{-\alpha(1 - 1/K)} + \sum\limits_{i: \tilde{h}(x_i, y_i) \leq 0}^n w_{t, i} e^{\alpha/K}} && \\
\end{flalign*}
Това, от което се интересуваме, е производната по $\alpha$.

\begin{tabular}{l c l}
	$\cfrac{\partial\B{\sum\limits_{i: \tilde{h}(x_i, y_i) > 0}^n w_{t, i} e^{-\alpha(1 - 1/K)} + \sum\limits_{i: \tilde{h}(x_i, y_i) \leq 0}^n w_{t, i} e^{\alpha/K}}}{\partial \alpha} = 0$ & & $\longleftrightarrow$ \\
	\\
	$-\alpha\B{1 - \cfrac{1}{K}}e^{-\alpha(1 - 1/K)}\B{\sum\limits_{i:\tilde{h}(x_i, y_i) > 0} w_{t, i}} + \cfrac{\alpha}{K}\ e^{\alpha/K}\B{\sum\limits_{i:\tilde{h}(x_i, y_i) \leq 0} w_{t, i}} = 0$ & & $\longleftrightarrow$\\
	\\
	$\cfrac{\alpha}{K}\ e^{\alpha/K}\B{\sum\limits_{i:\tilde{h}(x_i, y_i) \leq 0} w_{t, i}} = \alpha\B{1 - \cfrac{1}{K}}e^{-\alpha(1 - 1/K)}\B{\sum\limits_{i:\tilde{h}(x_i, y_i) > 0} w_{t, i}}$ & &$\longleftrightarrow$\\
	\\
	$\cfrac{\cancel{\alpha}e^{(\alpha/K + \alpha(1 - 1/K))}}{\cancel{K}}\cfrac{\cancel{K}}{\cancel{\alpha}(K - 1)} = \cfrac{\sum\limits_{i:\tilde{h}(x_i, y_i) > 0} w_{t, i}}{\sum\limits_{i:\tilde{h}(x_i, y_i) \leq 0} w_{t, i}}$ & & $\longleftrightarrow$\\
	\\ 
	$\cfrac{e^{\alpha}}{K-1} = \cfrac{\sum\limits_{i:\tilde{h}(x_i, y_i) > 0} w_{t, i}}{\sum\limits_{i:\tilde{h}(x_i, y_i) \leq 0} w_{t, i}}$ & /логаритмуваме & $\longleftrightarrow$\\
	\\  
	$\alpha = ln\B{\cfrac{\sum\limits_{i:\tilde{h}(x_i, y_i) > 0} w_{t, i}}{\sum\limits_{i:\tilde{h}(x_i, y_i) \leq 0} w_{t, i}}} + ln(K - 1)$ & &$\longleftrightarrow$\\
	\\
	$\alpha = ln\B{\cfrac{\sum\limits_{i=1}^n w_{t, i} - \sum\limits_{i:\tilde{h}(x_i, y_i) \leq 0} w_{t, i}}{\sum\limits_{i:\tilde{h}(x_i, y_i) \leq 0} w_{t, i}}} + ln(K - 1)$ & & \\
	Умножаваме и делим на $\sum\limits_{i=1}^n w_{t, i}$ в сумата &  &$\longleftrightarrow$\\
	\\
	$\alpha = ln\B{\cfrac{\cfrac{\sum\limits_{i=1}^n w_{t, i} - \sum\limits_{i:\tilde{h}(x_i, y_i) \leq 0} w_{t, i}}{\sum\limits_{i=1}^n w_{t, i}}}{\cfrac{\sum\limits_{i:\tilde{h}(x_i, y_i) \leq 0} w_{t, i}}{\sum\limits_{i=1}^n w_{t, i}}} } + ln(K - 1)$ & & $\longleftrightarrow$\\
	\\
	$\alpha = ln \B{\cfrac{1 - \varepsilon_t}{\varepsilon_t}} + ln(K-1)$
\end{tabular}

Което съвпада с избора на $\alpha$ от точка 3.3 на алгоритъма.
\end{proof}

От \autoref{appendix:ada:lemma:2} и \autoref{appendix:ada:lemma:3} следва, че полученото в точка 5. от алгоритъма $H$ ще е с минимална експоненциална грешка върху тренировъчните данни.

{\footnotesize \textbf{Наблюдение:} На стъпка} {\normalsize $argmin_{h\in \mathcal{H}} \B{ (e^{C/K} - e^{-C(1 - 1/K)}) \sum\limits_{i: \tilde{h}(x_i, y_i) \leq 0}^n w_{t, i}}$} {\footnotesize при намиране на минимално $h$ за улеснение приехме, че знакът на} {\normalsize $(e^{C/K} - e^{-C(1 - 1/K)})\ $}{\footnotesize  е положителен, за да запазим задачата за минимум. За $C=\alpha$ това означава, че $\alpha > 0$ и тъй като $\alpha = ln\B{\cfrac{1 - \varepsilon}{\varepsilon}} + ln(K-1)$, оттук идва ограничението, че $\varepsilon < \cfrac{K-1}{K}$. Тоест, грешката на всеки от класификаторите не трябва да надхвърля $\cfrac{K-1}{K}$}.
\end{document}

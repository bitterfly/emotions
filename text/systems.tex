\documentclass[main.tex]{subfiles}

\begin{document}
\section{Представяне със системи}
Нека имаме чистия сигнал от глотиса $g[t]$. При преминаването му през вокалния тракт и устните, той
се променя, в следствие на различни фактори като турболенция, поглъщане, отрязяване, в следствие на което 
на изхода (устните), получаваме сигнала $y[n]$.

\begin{definition*}{(Система)}\\
Механизъм, който манипулира един или повече сигнали с някаква цел до
получаване на нов сигнал, се нарича система.

Обикновено в практическия свят се използват системи, чието действие е предварително известно (и желано). Такива системи
наричаме \textbf{филтри}. Филтрите обикновено изпълняват някаква точно определена манипулация върху сигнала, например
да премахват всички честоти под или над определа честота.
\end{definition*}

С $g[n] \mapsto y[n]$ ще бележим, че $y$ е отговорът на системата за вход $g$. В такъв случай системата, която ще разгледаме, е тази на вокалния тракт. Ще ни интересуват
следните няколко класа системи.

\begin{definition*}{(Линейна система)}\\
    Ако $x_1[n] \mapsto y_1[n]$ и $x_2[n] \mapsto y_2[n]$, то системата е линейна $\longleftrightarrow$

    $ax_1[n] + bx_2[n] \mapsto ay_1[n] + by_2[n]$ 
\end{definition*}


\begin{definition*}{(Времево-независима система)}\\
    Нека $x[n] \mapsto y[n]$. Тогава, ако за всяко $n_0: x[n - n_0] \mapsto y[n - n_0]$, то
    системата е времево-независима.
\end{definition*}

\begin{property}
\label{systems:periodicity}
Ако системата е времево-независима и сигналът $x$ е периодичен с период $N$,
то и изходът на системата $y$ е периодичен с период $N$:

$x[n] \mapsto y[n]$ и $x[n] = x[n+N] \implies x[n+N] \mapsto y[n]$. Но от времевата независимост знаем, че
$x[n+N] \mapsto y[n+N] \implies y[n] = y[n+N]$
\end{property}


Специален подклас на линейните, времево-независими системи, е класът на системите, удовлетворяващи диференчното уравнение от ред $N$ с константни коефициенти:
\begin{flalign}
    \label{eq:systems:1}
    & \sum\limits_{k=0}^{N} a_k y [n-k] = \sum\limits_{m=0}^{M}b_k x[n-m] &&
\end{flalign}

Вокалният тракт е времево-независима система, защото изходът $y[n]$ не зависи от момента
от време, а само от специфичната му конфигурация в текущия момент, т.е. положението на езика, устните,
зъбите. 
Нека предположим, че вокалният тракт е линейна, времево-независима система, която удовлетворява уравнение $\autoref{eq:systems:1}$, и да разгледаме свойствата.

Искаме да опишем как работи тази система. За момента знаем как ще реагира тя, ако ѝ подадем входен сигнал $g[n]$.  
Но вместо да разглеждаме отговора на системата за широк спектър от входни функции, ще е полезно да имаме характеризация,
която не зависи от входа.

Пъро да разгледаме входа по различен начин. Ако за всеки момент от време $n_0$ имаме импулси със сила $g[n_0]$, то можем да мислим за входния сигнал $g[n]$
като за сума от тези импулси. Тоест, нека имаме дискретния единичен импулс:

\begin{flalign*}
    &\delta[n] = \begin{cases}
    1, & n = 0\\
    0, & \text{иначе}\\
    \end{cases} &&
\end{flalign*}

Тогава можем да представим входния сигнал $g[n]$ като
\begin{flalign*}
    & g[n] = \sum\limits_{k=-\infty}^{\infty} g[k]\delta[n-k] &&
\end{flalign*}

Нека $\delta[n-k] \mapsto h_k[n]$. Тъй като системата е линейна, е изпълнено, че:
\begin{flalign}
    \label{eq:systems:2}
    & g[n] = \sum\limits_{k=-\infty}^{\infty}g[k]\delta[n-k] \mapsto \sum\limits_{k = -\infty}^{\infty}g[k]h_k[n] = y[n] &&
\end{flalign}

Времевата инвариантност ни казва, че ако $\delta[n] \mapsto h[n]$, то $\delta[n -k] \mapsto h[n-k]$, следователно
в случая на вокалния тракт $\autoref{eq:systems:2}$ има вида:

\begin{flalign}
    \label{eq:systems:3}
    &  y[n] = \sum\limits_{k = -\infty}^{\infty}g[k]h_k[n] = \sum\limits_{k = -\infty}^{\infty}g[k]h[n-k] &&, 
\end{flalign}

или записано като \hyperref[appendix:fourier:conv]{конволюция} $y[n] = (g \ast h)[n]$.

Ако разгледаме Фурие преобразуванията на $y, g, h$, които са съответно $\mathcal{Y}, \mathcal{G}, \mathcal{H}$, в $\mathcal{z} = e^{iw_k}$, получаваме:
\begin{flalign}
    \label{eq:systems:4}
    & \nonumber \mathcal{Y}(\mathcal{z}) = \mathcal{G}(\mathcal{z})\mathcal{H}(\mathcal{z}) &&\\
    & \mathcal{H}(\mathcal{z}) = \cfrac{\mathcal{Y}(\mathcal{z})}{\mathcal{G}(\mathcal{z})} &&
\end{flalign}
$\mathcal{H}$ се нарича предавателна функция за системата.

Да разгледаме фурие преобразуванието на $\autoref{eq:systems:1}$ за входен сигнал $g$.
\begin{flalign}
    \label{eq:systems:5}
    & \nonumber\Q{\sum\limits_{k=0}^{N}a_k\mathcal{z}^{-k}}\mathcal{Y}(\mathcal{z}) = \Q{\sum\limits_{m=0}^{M} b_m \mathcal{z}^{-m}}\mathcal{G}(\mathcal{z}) &&\\
    & \cfrac{\mathcal{Y}(\mathcal{z})}{\mathcal{G}(\mathcal{z})} = \cfrac{\sum\limits_{m=0}^{M} b_m \mathcal{z}^{-m}}{\sum\limits_{k=0}^{N} a_k \mathcal{z}^{-k}} &&
\end{flalign}

Когато заместим $\autoref{eq:systems:5}$ в $\autoref{eq:systems:4}$, получаваме
\begin{flalign}
    \label{eq:systems:6}
    & \mathcal{H}(\mathcal{z}) =  \cfrac{\sum\limits_{m=0}^{M} b_m \mathcal{z}^{-m}}{\sum\limits_{k=0}^{N} a_k \mathcal{z}^{-k}} &&
\end{flalign}

В \autoref{tubes} видяхме, че резултатния сигнал $\mathcal{Y}$, който се получава при изходите на системата,
има следния вид:

\begin{flalign*}
    \tag{\ref{eq:tubes:27}}
    & \mathcal{Y}(\mathcal{z}) = \mathcal{G}(\mathcal{z})\mathcal{V}(\mathcal{z})\mathcal{R}(\mathcal{z}) = \mathcal{G}(\mathcal{z}) \cfrac{\sum\limits_{m=0}^{M} b_m \mathcal{z}^{-m} }{\sum\limits_{k=0}^{K} a_k \mathcal{z}^{-k}} &&
\end{flalign*}

Това означава, че $\mathcal{V}(\mathcal{z})\mathcal{R}(\mathcal{z})$, всъщност описват предавателната функция на системата $g[n] \mapsto y[n]$, тоест $\mathcal{H}(\mathcal{z}
) = \mathcal{V}(\mathcal{z})\mathcal{R}(\mathcal{z})$.

Следователно, производството на реч се описва от системата $\mathcal{Y}(\mathcal{z}) = \mathcal{G}(\mathcal{z})\mathcal{H}(\mathcal{z})$,
а $\mathcal{H}$ съдържа информацията за вокалния тракт.
Характеристиките, които ще изберем, трябва да носят тази информация за вокалния тракт,
тоест трябва да отделят входния сигнал $\mathcal{G}$ от филтъра $\mathcal{H}$, извличайки иформацията
за подлежащата емоция, която се надяваме, че е кодирана в $\mathcal{H}$.

Изборът на характеристики е описан по-подробно в следващия раздел.
\end{document}
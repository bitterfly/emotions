\documentclass[main.tex]{subfiles}

\begin{document}
\section{Представяне със системи}
Нека имаме чистия сигнал от глотиса $g[t]$. При преминаването му през вокалния тракт и устните, той
се променя, в следствие на различни фактори като турболенция, поглъщане, отрязяване, в следствие на което 
на изхода (устните), получаваме сигналът $y[n]$.

\begin{definition*}{(Система)}\\
Механизъм, който манипулира един или повече сигнали с някаква цел до
получаване на нов сигнал, се нарича система

Обикновено в практическия свят се използват системи, чието действие е предварително известно (и желано). Такива системи
наричаме \textbf{филтри}. Филтрите обикновено изпълняват някаква точно определена манипулация върху сигнала, например
да премахват всички честоти под или над определа честота.
\end{definition*}

С $g[n] \mapsto y[n]$ ще бележим, че $y$ е отговорът на системата за вход $g$. В такъв случай, системата, която ще разгледаме, е тази на вокалния тракт. Ще ни интересуват
няколко класа системи.

\begin{definition*}{(Линейна система)}\\
    Ако $x_1[n] \mapsto y_1[n]$ и $x_2[n] \mapsto y_2[n]$, то системата е линейна $\longleftrightarrow$

    $ax_1[n] + bx_2[n] \mapsto ay_1[n] + by_2[n]$ 
\end{definition*}


\begin{definition*}{(Времево-независима система)}\\
    Нека $x[n] \mapsto y[n]$. Тогава, ако за всяко $n_0: x[n - n_0] \mapsto y[n - n_0]$, то
    системата е времево-независима.
\end{definition*}

Специален подклас на линейните, времево-независими системи, е класът на системите, удовлетворяващи диференчното уравнение от ред $N$ с константни коефициенти:
\begin{flalign}
    \label{eq:systems:1}
    & \sum\limits_{k=0}^{N} a_k y [n-k] = \sum\limits_{m=0}^{M}b_k x[n-m] &&
\end{flalign}


Вокалният тракт е времето независима система, защото изходът $y[n]$ не зависи от момента
от време, а само от специфичната му конфигурация в текущия момент, т.е. положението на езика, устните,
зъбите. 
Нека предположим, че вокалния тракт е линейна, времево-независима система, която удовлетворява уравнение $\autoref{eq:systems:1}$, и да разгледаме свойствата.

Искаме да опишем как работи тази система. За момента знаем как ще реагира тя, ако ѝ подадем входен сигнал $g[n]$.  
Но вместо да разглеждаме отговора на системата за широк спектър от входни функции, ще е полезно да имаме характеризация,
която не зависи от входа.

Пъро да разгледаме входа по различен начин. Ако за всеки момент от време $n_0$ имаме импулси със сила $g[n_0]$, то можем да мислим за входния сигнал $g[n]$
като за сума от тези импулси. Тоест, нека имаме дискретният единичен импулс:

\begin{flalign*}
    &\delta[n] = \begin{cases}
    1, & n = 0\\
    0, & \text{иначе}\\
    \end{cases} &&
\end{flalign*}

Тогава можем да представим входния сигнал $g[n]$ като
\begin{flalign*}
    & g[n] = \sum\limits_{k=-\infty}^{\infty} g[k]\delta[n-k] &&
\end{flalign*}

Нека $\delta[n-k] \mapsto h_k[n]$. Тъй като системата е линейна, е изпълнено че:
\begin{flalign}
    \label{eq:systems:2}
    & g[n] = \sum\limits_{k=-\infty}^{\infty}g[k]\delta[n-k] \mapsto \sum\limits_{k = -\infty}^{\infty}g[k]h_k[n] = y[n] &&, 
\end{flalign}

Времевата инвариантност ни казва, че ако $\delta[n] \mapsto h[n]$, то $\delta[n -k] \mapsto h[n-k]$, следователно
в случая на вокалния тракт $\autoref{eq:systems:2}$ има вида:

\begin{flalign}
    \label{eq:systems:3}
    &  y[n] = \sum\limits_{k = -\infty}^{\infty}g[k]h_k[n] = \sum\limits_{k = -\infty}^{\infty}g[k]h[n-k] &&, 
\end{flalign}

\begin{definition*}{(Дискретна конволюция)}\\
Ако $f, g: \mathbb{N} \mapsto \mathbb{Z}$, дискретна конволюция (конволюционна сума) на $f$ и $g$, наричаме
$(f\ast g)[n] = \sum\limits_{k=-\infty}^{\infty} f[k]g[n-k]$
\end{definition*}

Тоест $y[n] = (g \ast h)[n]$
Ако запишем това равенство в $\mathcal{z}$-домейна, получаваме
\begin{flalign}
    \label{eq:systems:4}
    & \nonumber \mathcal{Y}(\mathcal{z}) = \mathcal{G}(\mathcal{z})\mathcal{H}(\mathcal{z}) &&\\
    & \mathcal{H}(\mathcal{z}) = \cfrac{\mathcal{Y}(\mathcal{z})}{\mathcal{G}(\mathcal{z})} &&
\end{flalign}
$\mathcal{H}$ се нарича предавателна функция за системата.

Да разгледаме $\autoref{eq:systems:1}$ в $\mathcal{z}$-домейна.
\begin{flalign}
    \label{eq:systems:5}
    & \nonumber\Q{\sum\limits_{k=0}^{N}a_k\mathcal{z}^{-k}}\mathcal{Y}(\mathcal{z}) = \Q{\sum\limits_{m=0}^{M} b_m \mathcal{z}^{-m}}\mathcal{G}(\mathcal{z}) &&\\
    & \cfrac{\mathcal{Y}(\mathcal{z})}{\mathcal{G}(\mathcal{z})} = \cfrac{\sum\limits_{m=0}^{M} b_m \mathcal{z}^{-m}}{\sum\limits_{k=0}^{N} a_k \mathcal{z}^{-k}} &&
\end{flalign}

Когато заместим $\autoref{eq:systems:5}$ в $\autoref{eq:systems:4}$, получаваме
\begin{flalign}
    \label{eq:systems:6}
    & \mathcal{H}(\mathcal{z}) =  \cfrac{\sum\limits_{m=0}^{M} b_m \mathcal{z}^{-m}}{\sum\limits_{k=0}^{N} a_k \mathcal{z}^{-k}} &&
\end{flalign}
\end{document}
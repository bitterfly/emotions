\documentclass[main.tex]{subfiles}
\begin{document}
\chapter{Приложение към \nameref{chap:speech}}

\begin{theorem}
\label{appendix:2:01}
$\mathcal{V}(\mathcal{z})$ за $N=2$ и произволно $\tau_i$
\end{theorem}

Имаме:

$U_k = Q_k U_{k+1}$ за

\begin{multicols}{2}
    $U_k = 
    \begin{bmatrix}
        U_k^{+}(\mathcal{z}) \\
        U_k^{-}(\mathcal{z}) \\
    \end{bmatrix}$
    \begin{flalign*}
        & Q_k = \left[\begin{array}{cc}
                \cfrac{\mathcal{z}^{\tau_k}}{1 + r_k} & \cfrac{-r_k \mathcal{z}^{\tau_k}}{1 + r_k} \\
                & \\
                \cfrac{-r_k \mathcal{z}^{-\tau_k}}{1 + r_k} & \cfrac{\mathcal{z}^{-\tau_k}}{1 + r_k} \\
            \end{array}\right] = 
            z^{\tau_k}\left[\begin{array}{cc}
                \cfrac{1}{1 + r_k} & \cfrac{-r_k}{1 + r_k} \\
                & \\
                \cfrac{-r_k \mathcal{z}^{-2\tau_k}}{1 + r_k} & \cfrac{\mathcal{z}^{-2\tau_k}}{1 + r_k} \\
            \end{array}\right] = 
            \mathcal{z}^{\tau_k} \hat{Q}_k && 
    \end{flalign*}
\end{multicols}

Търсим: $V(\mathcal{z})$
\begin{proof}[\unskip\nopunct]
    \begin{flalign}
        \tag{\ref{eq:tubes:22}}
        & \frac{1}{\mathcal{V}(\mathcal{z})} = \frac{U_G(\mathcal{z})}{U_L(\mathcal{z})}  = \left[
            \begin{array}{cc}
                \cfrac{2}{1+r_G}, & -\cfrac{2r_G}{1 + r_G} \\
            \end{array}
            \right] \prod_{i=1}^{N} {Q_i} \left[ \begin{array}{cc}
                1 \\
                0
            \end{array}\right] = &&
        \end{flalign}
        \begin{flalign*}
            & = \mathcal{z}^{(\tau_1 + \tau_2)}\left[
                \begin{array}{cc}
                    \cfrac{2}{1+r_G}, & -\cfrac{2r_G}{1 + r_G} \\
                \end{array}
                \right] {\hat{Q}_1}{\hat{Q}_2} \left[ \begin{array}{cc}
                    1 \\
                    0
                \end{array}\right] = && \\
            & \\
            & = \mathcal{z}^{(\tau_1 + \tau_2)}\left[
                \begin{array}{cc}
                    \cfrac{2}{1+r_G}, & -\cfrac{2r_G}{1 + r_G} \\
                \end{array}
                \right]
                \left[\begin{array}{cc}
                    \cfrac{1}{1 + r_1} & \cfrac{-r_1}{1 + r_1} \\
                    & \\
                    \cfrac{-r_1 \mathcal{z}^{-
                    2\tau_1}}{1 + r_1} & \cfrac{\mathcal{z}^{-2\tau_1}}{1 + r_1} \\
                \end{array}\right]
                \left[\begin{array}{cc}
                    \cfrac{1}{1 + r_2} & \cfrac{-r_2}{1 + r_2} \\
                    & \\
                    \cfrac{-r_2 \mathcal{z}^{-2\tau_2}}{1 + r_2} & \cfrac{\mathcal{z}^{-2\tau_2}}{1 + r_2} \\
                \end{array}\right]
                \left[ \begin{array}{cc}
                    1 \\
                    0
                \end{array}\right] = && \\
            &\\
            & = 2\mathcal{z}^{(\tau_1 + \tau_2)}
            \left[
                \begin{array}{cc}
                    \cfrac{1 + r_G r_1\mathcal{z}^{-2\tau_1}}{(1+r_G)(1 + r_1)}, & -\cfrac{r_1 + r_G\mathcal{z}^{-2\tau_1}}{(1 + r_G)(1 + r_1)} \\
                \end{array}
            \right]
            \left[\begin{array}{cc}
                \cfrac{1}{1 + r_2} & \cfrac{-r_2}{1 + r_2} \\
                & \\
                \cfrac{-r_2 \mathcal{z}^{-2\tau_2}}{1 + r_2} & \cfrac{\mathcal{z}^{-2\tau_2}}{1 + r_2} \\
            \end{array}\right]
            \left[ \begin{array}{cc}
                1 \\
                0
            \end{array}\right] = && \\
            &\\
            & = 2\mathcal{z}^{(\tau_1 + \tau_2)}
            \left[
                \begin{array}{cc}
                    \cfrac{1 + r_Gr_1\mathcal{z}^{-2\tau_1} + r_1r_2\mathcal{z}^{-2\tau_2} + r_Gr_2\mathcal{z}^{-2(\tau_1 + \tau_2)}}{(1+r_G)(1 + r_1)(1 + r_2)},& -\cfrac{r_2 + r_Gr_1r_2\mathcal{z}^{-2\tau_1} + r_1\mathcal{z}^{-2\tau_2} + r_G\mathcal{z}^{-2(\tau_1 + \tau_2)}}{(1 + r_G)(1 + r_1)(1 + r_2)} \\
                \end{array}
            \right]
            \left[ \begin{array}{cc}
                1 \\
                0
            \end{array}\right]&& \\
        \end{flalign*}
    \begin{flalign*}
        & \iff && \\
        & \mathcal{V}(\mathcal{z}) = \cfrac{0.5\mathcal{z}^{-(\tau_1 + \tau_2)}(1+r_G)\prod\limits_{i=1}^{2}(1 + r_i)}{1 + r_Gr_1\mathcal{z}^{-2\tau_1} + r_1r_2\mathcal{z}^{-2\tau_2} + r_Gr_2\mathcal{z}^{-2(\tau_1 + \tau_2)}} &&
    \end{flalign*}
\end{proof}

\hrulefill

\begin{footnotesize} \textbf{Бележка:}
\label{can:i}
Тък като $r_G, r_1, r_2$ са ненулеви, то за да получим максимална степен, без да имаме нулеви коефициенти, трябва:

\begin{flalign}
  & \begin{array}{|l@{}}
    2 \tau_1 = 1\\
    2 \tau_2 = 1\\
    2(\tau_1 + \tau_2) = 2
  \end{array} &&
\end{flalign}

Тоест $\tau_1 = \tau_2 = \frac{1}{2}$
\end{footnotesize}

\hrulefill

\begin{theorem}
    \label{appendix:2:02}
    $\mathcal{V}(\mathcal{z})$ за $N=2$ и произволно $\tau_1 = \tau_2 = \frac{1}{2}$
\end{theorem}

\begin{proof}[\unskip\nopunct]
Заместваме $\tau_i = \frac{1}{2}$ в \autoref{appendix:2:01}
\begin{flalign*}
    & \mathcal{V}(\mathcal{z}) = \cfrac{0.5\mathcal{z}^{-1}(1+r_G)\prod\limits_{i=1}^{2}(1 + r_i)}{1 + (r_Gr_1 + r_1r_2)\mathcal{z}^{-1} + r_Gr_2\mathcal{z}^{-2}} &&
\end{flalign*}
\end{proof}
    

\end{document}
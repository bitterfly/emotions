\documentclass[main.tex]{subfiles}
\begin{document}
\chapter{Приложение към \nameref{chap:em}}
\label{appendix:em}

\begin{theorem}
\label{appendix:em:th1}
$\frac{\partial x^T A x}{\partial x} = 2Ax$, ако $x \in \mathbb{R}^m, A\in \mathbb{R}^m\times\mathbb{R}^m$, а $A$ е диагонална.

Да разгледаме производната по някоя от координатите - $x_k$
\begin{flalign*}
    \frac{\partial x^T A x}{\partial x_k}  & = \frac{
        \partial \B{x_1, x_2, \dotso x_m}
        \begin{pmatrix}
            a_{11} & 0 & \dots & 0 \\
            0 & a_{22} & \dots & 0 \\
            \vdots & \vdots & \ddots & \vdots \\
            0 & 0 & \dots & a_{mm}
        \end{pmatrix}
        \begin{pmatrix}
            x_{1} \\
            x_{2} \\
            \vdots \\
            x_{m}
        \end{pmatrix}}{\partial x_k} & \\
        & = \frac{\partial\B{x_1 a_{11}, x_2 a_{22}, \dotso x_m a_{mm}}
        \begin{pmatrix}
            x_{1}\\
            x_{2}\\
            \vdots \\
            x_{m}
        \end{pmatrix}}{\partial x_k} & \\
        & = \frac{\partial \B{x_1^2 a_{11} + x_2^2 a_{22} + \dotso + x_m ^ 2 a_{mm} }}{\partial x_k} = 2x_k a_{kk}
\end{flalign*} 

Това означава, че:
\begin{flalign*}
    & \frac{\partial x^T A x}{\partial x} = \B{2x_1 a_{11}, 2x_2 a_{22} \dotso 2x_m a_{mm}} = 2 A x &
\end{flalign*}
\end{theorem}

\begin{theorem}
\label{appendix:em:th2}
Ако $A$ е диагонална матрица, $A=(a_{ii})_{i=1}^m, \cfrac{\partial |A|}{\partial a_{ii}} = \cfrac{|A|}{a_{ii}}$


$\cfrac{\partial |A|}{\partial a_{ii}} = \cfrac{\partial \B{\prod\limits_{i=1}^m a_{ii}}}{\partial a_{ii}} = a_{11}.a_{22}\dotso a_{i-1 i-1}a_{i+1 i+1}\dotso a_{mm} = \cfrac{\prod\limits_{i=1}^m a_{ii}}{a_{ii}} = \cfrac{|A|}{a_{ii}}$
\end{theorem}


Нека $L(\pi, \mu, \Sigma) = \sum\limits_{i=1}^{n} \log(\sum\limits_{k=1}^{K} \pi_k \mathcal{N}(x_i\mid \mu_k, \Sigma_k))) + \lambda(\sum\limits_{k=1}^K \pi_k - 1)$
и 

$\mathcal{N}(x_i, \mu_j, \Sigma_j) = \cfrac{\exp\B{-\frac{1}{2}(x_i - \mu_j)^{T}\Sigma^{-1}_j(x_i - \mu_j)}}{\sqrt{(2\pi)^{39}|\Sigma_j|}}$, 

\begin{lemma}
    Решението на $\cfrac{\partial L(\pi, \mu, \Sigma)}{\partial \mu_j} = 0$ има вида $\mu_j = \cfrac{\sum\limits_{i=1}^N \gamma_{ij}x_i}{\sum\limits_{i=1}^N \gamma_{ij}}$
\end{lemma}

\begin{proof}


$0 = \cfrac{\partial L(\pi, \mu, \Sigma)}{\partial \mu_j} = \sum\limits_{i=1}^{n} \Q{ \cfrac{\pi_j \cfrac{\partial \mathcal{N}(x_i, \mu_j, \Sigma_j)}{\partial \mu_j} }{\sum\limits_{k=1}^K \pi_k \mathcal{N}(x_i, \mu_k, \Sigma_k)}}$ 

Използвайки \autoref{appendix:em:th1}, можем да намерим производната на $\mathcal{N}(x_i, \mu_j, \Sigma_j)$ по $\mu_j$:

\begin{flalign*}
    \cfrac{\partial \mathcal{N}(x_i, \mu_j, \Sigma_j)}{\partial \mu_j} & = \partial\Q{\frac{\exp\B{-\frac{1}{2}(x_i - \mu_j)^{T}\Sigma^{-1}_j(x_i - \mu_j)}}{\sqrt{(2\pi)^{39}|\Sigma_j|}}}/\partial \mu_j & \\
    & = \frac{\exp\B{-\frac{1}{2}(x_i - \mu_j)^{T}\Sigma^{-1}_j(x_i - \mu_j)}}{\sqrt{(2\pi)^{39}|\Sigma_j|}} \B{-\frac{1}{2}  2 \Sigma_j^{-1} (x_i - \mu_j)(-1)} & \\
    & = \mathcal{N}(x_i, \mu_j, \Sigma_j)\Sigma_j^{-1}(x_i - \mu_j) &
\end{flalign*}

Следователно:
\begin{flalign*}
    & 0 = \cfrac{\partial L(\pi, \mu, \Sigma)}{\partial \mu_j} = \sum\limits_{i=1}^{n} \Q{ \cfrac{\pi_j \cfrac{\partial \mathcal{N}(x_i, \mu_j, \Sigma_j)}{\partial \mu_j} }{\sum\limits_{k=1}^K \pi_k \mathcal{N}(x_i, \mu_k, \Sigma_k)}} = \sum\limits_{i=1}^{n} \Q{\cfrac{\pi_j \mathcal{N}(x_i, \mu_j, \Sigma_j)\Sigma_j^{-1}(x_i - \mu_j)}{\sum\limits_{k=1}^K \pi_k \mathcal{N}(x_i, \mu_k, \Sigma_k)}} & \\
    & \longleftrightarrow & \\
    & \sum\limits_{i=1}^N\Q{ \cfrac{\pi_j \mathcal{N}(x_i, \mu_j, \Sigma_j)\cancel{\Sigma_j^{-1}}x_i}{\sum\limits_{k=1}^K \pi_k \mathcal{N}(x_i, \mu_k, \Sigma_k)} } = \sum\limits_{i=1}^N\Q{ \cfrac{\pi_j \mathcal{N}(x_i, \mu_j, \Sigma_j)\cancel{\Sigma_j^{-1}}\mu_j}{\sum\limits_{k=1}^K \pi_k \mathcal{N}(x_i, \mu_k, \Sigma_k)} }& 
\end{flalign*}

Нека означим $\gamma_{ij} = \cfrac{\pi_j \mathcal{N}(x_i, \mu_j, \Sigma_j)}{\sum\limits_{k=1}^K \pi_k \mathcal{N}(x_i, \mu_k, \Sigma_k)}$. Тогава имаме:

\begin{flalign*}
    & \sum\limits_{i=1}^N \gamma_{ij}x_i = \mu_j \sum\limits_{i=1}^N \gamma_{ij}& \\
    & \mu_j = \cfrac{\sum\limits_{i=1}^N \gamma_{ij}x_i}{\sum\limits_{i=1}^N \gamma_{ij}}  &
\end{flalign*}

\end{proof}
\hrulefill

\begin{lemma}
    Решението на $\cfrac{\partial L(\pi, \mu, \Sigma)}{\partial \Sigma_j} = 0$ има вида $\Sigma_j = \begin{cases}
        \cfrac{\sum\limits_{i=1}^N \gamma_{ij} (x_{it} - \mu_{js})^2}{\sum\limits_{i=1}^N \gamma_{ij}}, & t==s \\
        0, & \text{иначе}
    \end{cases}$
\end{lemma}

\begin{proof}
\begin{flalign*}
    & 0 = \cfrac{\partial L(\pi, \mu, \Sigma)}{\partial \Sigma_{j}} = \sum\limits_{i=1}^n \Q{ \cfrac{\pi_j \cfrac{\partial\mathcal{N}(x_i, \mu_j, \Sigma_j)}{\partial \Sigma_j}}{ \sum\limits_{k=1}^K \pi_k \mathcal{N}(x_i, \mu_k, \Sigma_k) }} & 
\end{flalign*}

$\Sigma_j = (\sigma_{ij})_{m\times m}$ и $\sigma_{ij} = 0$, ако $i\neq j$. 
Първо смятаме:
\begin{flalign*}
    & \cfrac{\partial\Q{(x_i - \mu_j)^T\Sigma_j^{-1}(x_i - \mu_j)}}{\partial \sigma_{ts}} = & \\
    & = \cfrac{\partial\B{\B{(x_{i1} - \mu_{j1}), (x_{i2} - \mu_{j2}), \dotso (x_{im} - \mu_{jm})}
    \begin{pmatrix}
        \cfrac{1}{\sigma_{11}} & 0 & \dots & 0 \\
        0 & \cfrac{1}{\sigma_{22}} & \dots & 0 \\
        \vdots & \vdots & \ddots & \vdots \\
        0 & 0 & \dots & \cfrac{1}{\sigma_{mm}}
    \end{pmatrix}
    \begin{pmatrix}
        (x_{i1} - \mu_{j1}) \\
        (x_{i2} - \mu_{j2}) \\
        \vdots\\
        (x_{im} - \mu_{jm})
    \end{pmatrix} }}{\partial \sigma_{ts}} = & \\
    & = \cfrac{\partial\B{\cfrac{(x_{i1} - \mu_{j1})^2}{\sigma_{11}} + \cfrac{(x_{i2} - \mu_{j2})^2}{\sigma_{22}} + \dotso + \cfrac{(x_{im} - \mu_{jm})^2}{\sigma_{mm}} }}{\partial \sigma_{ts}} = & \\
    & = \begin{cases}
        0, & t \neq s \\
        \cfrac{-(x_{it} - \mu_{jt})^2}{\sigma_{tt}^2}, & t = s
    \end{cases} &
\end{flalign*}

Да разгледаме производната по произволен елемент $\sigma_{ts}$:
\begin{flalign*}
    & \cfrac{\partial \mathcal{N}(x_i, \mu_j, \Sigma_j)}{\partial \sigma_{ts}} = \cfrac{\partial \Q{\cfrac{\exp\B{-\frac{1}{2}(x_i - \mu_j)^{T}\Sigma^{-1}_j(x_i - \mu_t)}}{\sqrt{(2\pi)^{39}|\Sigma_j|}}}}{\partial \sigma_{ts}} & \\
    & = \cfrac{\cfrac{\delta\Q{\exp\B{-\frac{1}{2}(x_i - \mu_j)^{T}\Sigma^{-1}_j(x_i - \mu_t)}} }{\partial \sigma_{ts}}\sqrt{(2\pi)^{39}|\Sigma_j|} - \exp\B{-\frac{1}{2}(x_i - \mu_j)^{T}\Sigma^{-1}_j(x_i - \mu_t)} \cfrac{\delta\Q{ \sqrt{(2\pi)^{39}|\Sigma_j|} } }{\partial \sigma_{ts}}}{\B{\sqrt{(2\pi)^{39}|\Sigma_j|}}^2} & \\
    & = \quad\cfrac{ \exp\B{-\frac{1}{2}(x_i - \mu_j)^{T}\Sigma^{-1}_j(x_i - \mu_t)}\frac{1}{2}\cfrac{(x_{it} - \mu_{jt})^2}{\sigma_{tt}^2} \sqrt{(2\pi)^{39}|\Sigma_j|} }{\B{\sqrt{(2\pi)^{39}|\Sigma_j|}}^2} & \\
    & \quad -\cfrac{\exp\B{-\frac{1}{2}(x_i - \mu_j)^{T}\Sigma^{-1}_j(x_i - \mu_t)}\frac{1}{2}\cfrac{(2\pi)^{39}|\Sigma_j|}{ \sqrt{(2\pi)^{39}|\Sigma_j|} \sigma_{ts} }}{\B{\sqrt{(2\pi)^{39}|\Sigma_j|}}^2} = & \\
    & \cfrac{\mathcal{N}(x_i, \mu_j, \Sigma_j)(x_{it} - \mu_{jt})^2}{2\sigma_{tt}^2} - \cfrac{\mathcal{N}(x_i, \mu_j, \Sigma_j)}{2\sigma_{tt}} = \cfrac{\mathcal{N}(x_i, \mu_j, \Sigma_j)}{2\sigma_{tt}^2}\Q{(x_{it} - \mu_{jt})^2 - \sigma_{tt}}  &
\end{flalign*}

\begin{flalign*}
    & 0 = \cfrac{\partial L(\pi, \mu, \Sigma)}{\partial \sigma^j_{tt}} = \sum\limits_{i=1}^n \Q{ \cfrac{\pi_j \cfrac{\partial\mathcal{N}(x_i, \mu_j, \Sigma_j)}{\partial \sigma^j_{tt}}}{ \sum\limits_{k=1}^K \pi_k \mathcal{N}(x_i, \mu_k, \Sigma_k) }} & \\\\
    & = \sum\limits_{i=1}^N \cfrac{\pi_j \mathcal{N}(x_i, \mu_j, \Sigma_j) \Q{(x_{it} - \mu_{jt})^2 - \sigma_{tt}}}{2\sigma_{tt}^2 \sum\limits_{k=1}^K \pi_k \mathcal{N}(x_i, \mu_k, \Sigma_k)}  & \\\\
    & \longleftrightarrow & \\\\
    & \sum\limits_{i=1}^N \cfrac{\pi_j \mathcal{N}(x_i, \mu_j, \Sigma_j)(x_{it} - \mu_{jt})^2}{\cancel{2\sigma_{tt}^2} \sum\limits_{k=1}^K \pi_k \mathcal{N}(x_i, \mu_k, \Sigma_k)} = \sum\limits_{i=1}^N \cfrac{\pi_j \mathcal{N}(x_i, \mu_j, \Sigma_j) \sigma_{tt}}{\cancel{2\sigma_{tt}^2} \sum\limits_{k=1}^K \pi_k \mathcal{N}(x_i, \mu_k, \Sigma_k)} & \\\\
    & \sum\limits_{i=1}^N \gamma_{ij} (x_{it} - \mu_{jt})^2 = \sigma_{tt}\sum\limits_{i=1}^N \gamma_{ij} & \\\\
    & \sigma_{tt} = \cfrac{\sum\limits_{i=1}^N \gamma_{ij} (x_{it} - \mu_{jt})^2}{\sum\limits_{i=1}^N \gamma_{ij}} & \\\\
    & \Sigma_j = 
    \begin{pmatrix}
        \cfrac{\sum\limits_{i=1}^N \gamma_{ij} (x_{i1} - \mu_{j1})^2}{\sum\limits_{i=1}^N \gamma_{ij}} & 0 & \dots & 0 \\
        0 & \cfrac{\sum\limits_{i=1}^N \gamma_{ij} (x_{i2} - \mu_{j2})^2}{\sum\limits_{i=1}^N \gamma_{ij}} & \dots & 0 \\
        \vdots & \vdots & \ddots & \vdots \\
        0 & 0 & \dots & \cfrac{\sum\limits_{i=1}^N \gamma_{ij} (x_{im} - \mu_{jm})^2}{\sum\limits_{i=1}^N \gamma_{ij}}
    \end{pmatrix} & \\\\
    & = \begin{cases}
        \cfrac{\sum\limits_{i=1}^N \gamma_{ij} (x_{it} - \mu_{js})^2}{\sum\limits_{i=1}^N \gamma_{ij}}, & t==s \\
        0, & \text{иначе}
    \end{cases} &
\end{flalign*}
\end{proof}
\hrulefill
\begin{lemma}
    Решението на $\cfrac{\partial L(\pi, \mu, \Sigma)}{\partial \pi_j} = 0$ има вида $\pi_j = \cfrac{\sum\limits_{i=1}^N \gamma_{ij}}{N}$
\end{lemma}
\begin{proof}
\begin{flalign*}
    0 & = \cfrac{\partial L(\pi, \mu, \Sigma)}{\partial \pi_j} = \cfrac{\partial \Q{\sum\limits_{i=1}^{n} \log(\sum\limits_{k=1}^{K} \pi_k \mathcal{N}(x_i\mid \mu_k, \Sigma_k))) + \lambda(\sum\limits_{k=1}^K \pi_k - 1)}}{\partial \pi_j} = & \\
    & = \sum\limits_{i=1}^N  \Q{\frac{\mathcal{N}(x_i, \mu_j, \Sigma_j)}{\sum\limits_{k=1}^K \pi_k \mathcal{N}(x_i, \mu_k, \Sigma_k)}} + \lambda = \pi_j \B{\sum\limits_{i=1}^N  \Q{\frac{\mathcal{N}(x_i, \mu_j, \Sigma_j)}{\sum\limits_{k=1}^K \pi_k \mathcal{N}(x_i, \mu_k, \Sigma_k)}} + \lambda} &\\
    & \longleftrightarrow & \\
    & -\lambda\pi_j = \pi_j \sum\limits_{i=1}^N  \Q{\frac{\mathcal{N}(x_i, \mu_j, \Sigma_j)}{\sum\limits_{k=1}^K \pi_k \mathcal{N}(x_i, \mu_k, \Sigma_k)}} & \\
    & -\lambda\sum_{j=1}^K \pi_j = \sum\limits_{j=1}^K \pi_j \sum\limits_{i=1}^N  \Q{\frac{\mathcal{N}(x_i, \mu_j, \Sigma_j)}{\sum\limits_{k=1}^K \pi_k \mathcal{N}(x_i, \mu_k, \Sigma_k)}} &
\end{flalign*}
Използваме, че $\sum\limits_{j=1}^K \pi_j = 1$ и разменяме местата на сумите в дясно:
\begin{flalign*}
    & - \lambda = \sum\limits_{i=1}^N\Q{\cfrac{\sum\limits_{j=1}^K \pi_j \mathcal{N}(x_i, \mu_j, \Sigma_j)}{\sum\limits_{k=1}^K \pi_k \mathcal{N}(x_i, \mu_k, \Sigma_k)} } = \sum\limits_{i=1}^N 1& \\
    & \lambda = -N &
\end{flalign*}
Заместваме $\lambda = -N$ в 
\begin{flalign*}
    -\lambda\pi_j & = \pi_j \sum\limits_{i=1}^N  \Q{\frac{\mathcal{N}(x_i, \mu_j, \Sigma_j)}{\sum\limits_{k=1}^K \pi_k \mathcal{N}(x_i, \mu_k, \Sigma_k)}} & \\
    N\pi_j & = \sum\limits_{i=1}^N  \Q{\frac{\pi_j \mathcal{N}(x_i, \mu_j, \Sigma_j)}{\sum\limits_{k=1}^K \pi_k \mathcal{N}(x_i, \mu_k, \Sigma_k)}} & \\
    N\pi_j & = \sum\limits_{i=1}^N \gamma_{ij} \longleftrightarrow \pi_j = \cfrac{\sum\limits_{i=1}^N \gamma_{ij}}{N} &\\ 
\end{flalign*}
\end{proof}

\end{document}

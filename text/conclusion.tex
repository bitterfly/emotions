\documentclass[main.tex]{subfiles}
\begin{document}
Ако разглобиш котка, за да видиш как работи, първото нещо, което ще имаш в ръцете си, е неработеща котка\footnote{Последен цитат от Дъглас Адамс}. Затова нека разглобим текста на дипломната работа, за да можем да направим някакво заключение.

Първо бе дадена \textbf{някаква} дефиниция за емоция, като израз на нашето безсилие с дефиницията на понятието, и бяха избрани четири основни емоции, които да разпознаваме - щастие, гняв, тъга и неутрална емоция.

След това беше разгледана обработката на двата сигнала поотделно. Дадена бе обосновка как извличаме характеристики от речевия сигнал и защо предполагаме, че те носят информация за емоцията, съдържаща се в него. Показани са резултати от класификацията с описания в \autoref{chap:em} класификатор върху няколко бази данни.

Следващата глава се занимава с обработката на ЕЕГ сигнала. След преглед как извличаме информация от електроенцефалографа и какви характеристики ще мерим в получения сигнал, текстът разглежда създаването на две бази данни за ЕЕГ сигнал. Оказа се, че тази част на дипломната работа всъщност е най-времеотнемаща и резултатната база данни не е никак задоволителна. И тук са показани класификационни резултати със същия класификатор.

Последната част от дипломната работа се занимава със съчетаването на двата сигнала. Разгледани са два метода. Първият, конкатенация на характеристичните вектори, води до по-ниски резултати спрямо по-добрия класификатор и е даден по-скоро за пълнота. Подобряване на този метод може евентуално да се получи след намаляне на пространството. Вторият метод на съчетаване използва тегла за двата класификатора, намерени чрез модел, максимизиращ ентропията. От резултатите се вижда, че няма допълнително подобрение на ниво файл и има минимално такова на ниво вектор. 

\textbf{Заключението, което можем да направим, е, че с тази постановка съчетаването на двата сигнала не довежда до подобрение.}

Оттук-насетне могат да се изкажат спекулации защо.\\
От една страна, резултатът може да се дължи на лошото качество на базата данни. Тя не е създадена чрез експертно мнение и, поради трудностите при работа с енцефалографа, е много малка.\\
От друга страна, може би просто е факт, че сигналът от реч не съдържа допълнително информация, спрямо този от енцефалографа. Всъщност, в \cite{synt} показват как може да се синтезира реч директно от мозъчните вълни. Има няколко нива, на които може да се извлече информацията - първо, може да се хване ``командата'', която се изпраща на мускулите на вокалния тракт. Второ, имайки достъп до всеки индивидуален неврон, може да се извлече \textbf{намерението} да се изпрати тази команда. Тогава при достатъчно точно измерване (което означава \textbf{много} повече от 19 електрода), няма как сигналът за реч да носи допълнително информация, тъй като всяко действие на тялото така или иначе идва от мозъка. За съжаление, имайки не много точен електроенцефалограф, не можем да твърдим, че случаят е такъв.

И двете области - извличане на информация от реч и извличане на информация от ЕЕГ сигнал - са изключително обширни, което значи, че темата на тази дипломна работа търпи допълнително развитие. Както достатъчно пъти ми казаха обаче, човек все някъде трябва да сложи точка.

{\footnotesize Сбогом и благодаря за рибата!\footnote{Излъгах за последния цитат}}
\end{document}

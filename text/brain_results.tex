\documentclass[main.tex]{subfiles}

\begin{document}
\section{Данни и резултати}
Наличните бази данни за ЕЕГ са много по-малко от тези за реч, тъй като направата им е доста по-трудна. Обикновено при записа на такива данни се цели да има възможно най-малко дразнители за субекта и заради това има и много по-строги изисквания за самата постановка. Например, субектите трябва да са седнали удобно на стола, да не им прекалено топло или студено, да са спокойни и сити. Тъй като на експеримета могат да повлияят всякакви дразнители, се подсигурява субектите да не са консумирали кофеин в последните 24 часа, да са добре наспани и други подобни. Често се съобщават подробности като дали субектът е десничар или левичар, дали е пушач, тъй като не се знае дали няма да повлияят на експеримента. Записи, при които субектът е мигнал, се трият.

Целта на тази дипломна работа, обаче, е да се изследва какво ще стане като комбинираме информацията за емоцията в речта и ЕЕГ сигнала. Това изисква да имаме такава постановка, при която човекът говори, докато е свързан към електроенцефалограф, за да имаме данни от ЕЕГ и реч от един и същи момент от време. Ако мигането е забранено, то комплексна дейност като говоренето е направо еретична за наличните бази данни. Поради тази причина, и поради липсата\footnote{до колкото ми е известно} на подходяща база, такава трябваше да бъде създадена за конкретните цели.

Идеята е да се съчетаят най-често използваните похвати за създаване на емоционална речева база от една страна и емоциална ЕЕГ база от друга.

За реч най-често се използва една от следните две постановки: субектите (най-често актьори) четат предварително избрани емоционални изречения; субектите разказват емоционална случка по свой избор.

За ЕЕГ най-често се използва един от следните стимули: гледане на специално подбрани емоционални картинки\footnote{Например \url{https://web.archive.org/save/https://csea.phhp.ufl.edu/media/iapsmessage.html}}; слушане на специално подбрана емоционална музика; гледане на специално избрани емоционалени клипчета.

Първият опит за емоционална база данни бе следният: гледат се 15 специално избрани картинки (5 положителни, 5 отрицателни, 5 неутрални), като между картинките се показва черен екран за 2 секунди. На екранът се показва обяснение за следващата част на експеримента и 10 секунди почивка. Следващата част на експеримента повтаря няколко пъти следния цикъл: гледа се произволно емоционално клипче; субектът описва какво е видял на клипчето в продължение на минута; 10 секунди пауза; субектът чете показаният му на екрана тест; 10 секунди пауза.
Клипчетата са 3 за щастие, 3 за тъга и 3 за гняв, като средната продължителност е 1 минута. Текстовете са подбрани от прогнозата за времето. Счита се, че докато човек чете прогнозата за времето, няма да се вълнува много и ще произведе неутрални данни.
Субектът предварително знае постановката на опита.

При тестване на първоначално произведените данни се вижда, че периодът от 1-2 минути, през които се гледа видео, е твърде кратък, за да може да се развие някаква емоционална реакция. За съжаление, използването на мокри електроди позволява максимална продължителност на целия експеримент около 10-15 минути, след което време соления разтвор по електродите изсъхва и съпротивлението Някой от емоциите (например ``тъга'') изискват по-дълго време, за да се изградят напълно, други са моменти (като щастие). Също така, за някой хора е трудно изцяло да изпитват дадена емоция, в следствие на визуален стимул, вероятно и не всички хора са способни да изпитват емпатия при подобна ситуация. Заради това вторият опит разчита на лични преживявания.

Вторият опит се отдалечава повече от стандартните постановки, тъй като използва множество различни дразнители. Той цели да предразположи максимално субекта, според неговите нужди.

\end{document}

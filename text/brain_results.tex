\documentclass[main.tex]{subfiles}

\begin{document}
\section{Данни и резултати}
Наличните бази данни за ЕЕГ са много по-малко от тези за реч, тъй като направата им е доста по-трудна. Обикновено при записа на такива данни се цели да има възможно най-малко дразнители за субекта и заради това има и много по-строги изисквания за самата постановка. Например, субектите трябва да са седнали удобно на стола, да не им прекалено топло или студено, да са спокойни и сити. Тъй като на експеримета могат да повлияят всякакви дразнители, се подсигурява субектите да не са консумирали кофеин в последните 24 часа, да са добре наспани и други подобни. Често се съобщават подробности като дали субектът е десничар или левичар, дали е пушач, тъй като не се знае дали няма да повлияят на експеримента. Записи, при които субектът е мигнал, се трият.
Целта на тази дипломна работа, обаче, е да изследва какво ще стане като комбинираме информацията за емоцията в речта и ЕЕГ сигнала. Това изисква да имаме такава постановка, при която човекът говори, докато е свързан към електроенцефалограф, за да имаме данни от ЕЕГ и реч от един и същи момент от време. Ако мигането е забранено, то комплексна дейност като говоренето е направо еретична за наличните бази данни. Поради тази причина, и поради липсата\footnote{до колкото ми е известно} на подходяща база, такава трябваше да бъде създадена за конкретните цели.
\end{document}

\documentclass[main.tex]{subfiles}

\begin{document}
\section{Данни и резултати}
Изборът на (свободни) емоционални бази данни за реч всъщност е доста богат. Проблемът произтича от това, че в областта рядко се прави опит за повтаряне на резултати и дори сравняване с чужди такива. Това прави поставянето на резултатите в перспектива изключително трудно. 

Един от най-често използваните източници е берлинската емоционална база данни Emo-DB \cite{berlin}. Изборът на специфично тази база данни е по-скоро за да може да се сравни резултатът, постигнат с гореописаните методи, отколкото заради някакво нейно преимущество (освен лесното сдобиване с нея). Emo-DB се състои от 800 записа, в които 10 актьора (5 мъже и 5 жени) изиграват 7 емоции, всяка от които представена с 10 изречения (и няколко допълнителни втори опита). От изразените емоции, избираме тези за гняв, тъга, щастие и неутрално състояние. Броят и дължините на наличните файлове са описани в \autoref{tab:speech:results:01}

\begin{table}[h]
    \begin{center}
    \begin{tabular}{|l|r|r|} 
        \hline
        Емоция & Брой файлове & Обща дължина\\ 
        \hline
        Гняв & 127 & 5 мин. 35 сек.\\ 
        Щастие &  71 & 3 мин. 00 сек.\\ 
        Неутрално състояние &  79 & 3 мин. 06 сек.\\ 
        Тъга &  61 & 4 мин. 05 сек.\\ 
        \hline
    \end{tabular}
    \caption{Дължина и брой файлове в emo-DB за гняв, тъга, щастие и неутрално състояние}
    \label{tab:speech:results:01}
    \end{center}
\end{table}

В таблица \autoref{tab:speech:results:02} са показани резултати върху берлинската база данни. Изследванията използват подобни характеристики.
\begin{table}[h]
\begin{center}
    \begin{tabular}{ |l|c|c|l| } 
     \hline
     Източник & Тип характеристични вектори & Тип класификатор & Резултат \\ 
     \hline
     Текущ & GMM & MFCC & 82.20\% \\ 
     \cite{first} & SVM & LPCMFCC & 82.50\% \\ 
     \cite{second} &  Naïve Bayes classifier & MFCC & 82.76\% \\ 
     \cite{third} &  Random forest & MFCC & 79.02\% \\ 
     \hline
    \end{tabular}
    \caption{Резултати върху emo-DB}
    \label{tab:speech:results:02}
    \end{center}
\end{table}

При разглеждане на матрицата на грешките, показана в \autoref{tab:speech:results:03}, се вижда, че класификаторът
бърка емоции със сходна енергия, макар те да имат различна валентност. Това е добре известен проблем при разпознаване на емоции от реч. Точно поради тази причина речевите данни често се съчетават с допълнителен източник като видео записи, например. 

\begin{table}[h]
    \begin{center}
    \begin{tabular}{|l|r r r r|} 
        \hline
        & Гняв & Щастие & Неутрално & Тъга \\ 
        \hline
        Гняв &  \textbf{91.67\%} & 7.50\% & 0.00\% & 0.01\% \\ 
        Щастие & 37.14\% & \textbf{54.29\%} & 7.14\% & 1.43\% \\ 
        Неутрално & 0.00\% & 1.43\% & \textbf{82.86\%} & 15.71\% \\ 
        Тъга & 0.00\% & 0.00\% & 0.00\% & \textbf{100.00\%}\\ 
        \hline
        Общо & & & & \textbf{82.20\%}\\
        \hline
    \end{tabular}
    \caption{Матрица на грешките за база данни emo-DB}
    \label{tab:speech:results:03}
    \end{center}
\end{table}

Втората база данни, която ще разглеждаме, е за български. Тя е компилирана (мъчително) за целите на тази дипломна работа. Състои се от записи от предаването ``Тази сутрин'' \cite{tazi-sutrin}, които са ръчно класифицирани в четирите емоционални категории. Използвани са записи от Януари 2019 година назад до първото налично видео на сайта. Размерите и броя файлове са изложени в \autoref{tab:speech:results:04}

\begin{table}[ht]
    \begin{center}
    \begin{tabular}{|l|r|r|} 
        \hline
        Емоция & Брой файлове & Обща дължина\\ 
        \hline
        Гняв & 51 & 4 мин. 15 сек.\\ 
        Щастие &  33 & 2 мин. 45 сек.\\ 
        Неутрално състояние &  18 & 1 мин. 30 сек.\\ 
        Тъга &  22 & 1 мин. 59 сек.\\ 
        \hline
    \end{tabular}
    \caption{Дължина и брой файлове в базата данни от ``Тази сутрин'' за гняв, тъга, щастие и неутрално състояние}
    \label{tab:speech:results:04}
    \end{center}
\end{table}
Матрицата на грешките е показана на \autoref{tab:speech:results:05}.
За съжаление е трудно да се сравнят берлинската и местната база данни. Първо, езикът и етническата принадлежност може би играят голяма роля в изразяването на емоцията. Второ, emo-DB е записана в контролирана среда от професионални актьори, докато записите от ``Тази сутрин'' са хаотично записани, а професионалните говорители са умишлено избягвани. При трениране върху берлинската бази данни и тестване с ``Тази сутрин'' се получава разпознаване от $39.69\%$, а резултат от $34.64\%$ се наблюдава при обратната конфигурация.
\begin{table}[h]
    \begin{center}
    \begin{tabular}{|l|r r r r|} 
        \hline
        & Гняв & Щастие & Неутрално & Тъга \\ 
        \hline
        Гняв &  \textbf{72.00\%} & 4.00\% & 12.00\% & 12.00\% \\ 
        Щастие & 10.00\% & \textbf{73.33\%} & 0.00\% & 16.67\% \\ 
        Неутрално & 6.67\% & 6.67\% & \textbf{67.67\%} & 20.00\% \\ 
        Тъга & 0.00\% & 0.00\% & 0.00\% & \textbf{100.00\%}\\ 
        \hline
        Общо & & & & \textbf{78.00\%}\\
        \hline
    \end{tabular}
    \caption{Матрица на грешките за база данни от ``Тази сутрин''}
    \label{tab:speech:results:05}
    \end{center}
\end{table}

\end{document}

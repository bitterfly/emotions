\documentclass[main.tex]{subfiles}

\begin{document}
\section{Класификация}

След като сме избрали характеристичните вектори, които ще извличаме по подаден файл, трябва да можем да ги класифицираме по някакъв начин.
В \cite{survey} са разгледани и сравнени различни методи за класификация. Макар че невронните мрежи са по-често използвани в последните публикации в областта,
тук ще се подходи по ,,старомодния'' начин с Гаусови смески. Показаното за тях съотношение между ,,прецизност на разпознаване'' и ,,време за трениране'' е най-добро спред проучването.

Целим да намерим разпределение за всяка от търсените емоции. Всяко непрекъснато разпределение може да се приближи с произволна точност със смеска от достатъчно на брой гаусиани. Нека сме го постигнали с $K$ на брой гаусиани. Тогава при подаден характеристичен вектор $x$, ще търсим смеската на коя емоция ще доведе до най-голямо правдоподобие - тоест параметрите на кой модел е най-вероятно да са генерирали наблюдението. При подадени $\mu_i^e, \Sigma_i^e, i=1...K$ за дадена емоция $e$ и характеристичен вектор $x$, правдоподобието се пресмята с формулата:

$p(x) = \sum\limits_{k=1}^{K} \pi_k \mathcal{N}(x\mid \mu_k, \Sigma_k)$


Нека$X=(x_1,...x_n)$ са характеристични вектори с етикет $e$ 
\end{document}

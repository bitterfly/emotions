\documentclass[main.tex]{subfiles}

\begin{document}
\section{Класификация}

След като сме избрали характеристичните вектори, които ще извличаме по подаден файл, трябва да можем да ги класифицираме по някакъв начин.
В \cite{survey} са разгледани и сравнени различни методи за класификация. Макар че невронните мрежи са по-често използвани в последните публикации в областта,
тук ще се подходи по ,,старомодния'' начин с Гаусови смески. Показаното за тях съотношение между ,,точност на разпознаване'' и ,,време за трениране'' е най-добро спред проучването.


\end{document}

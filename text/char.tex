\documentclass[main.tex]{subfiles}

\begin{document}
\section{Характеристики}
    \subsection{Избор}
    \subsection{Извличане}
        Първо се изчита wav файла, като данни се запазват в масив от float числа.
        Базирайки се на идеята, че вокалния тракт е статичен за много къс период от време,
        накъсваме масива с данните на отделни застъпващи се фреймове, в рамките на които сигналът е статичен ("представяме си, че е статичен"). За да получим добра
        добра честотна резолюция, трябва да се включат голям брой samples от файла, но колкото по-голяма е дължината
        на фрейма, толкова по-голям е шансът да включим данни за различни конфигурации на глотиса.
        За да се постигне някакъв trade-off между двете, обикновено стойностите, които се избират, са 
        25 милисекунди за дължина на фрейм и 10 милисекунди за разстояние между два последователни фрейма.[Paul]
        
        Тъй като алгоритъмът за извличане на mfcc коефициенти, изисква броя на самплите да е степен на 2,
        допълваме последните фреймове с 0, ако това е нужно, тъй като това не влияе на точността. [мжое би онази статия за fourier]
\end{document}